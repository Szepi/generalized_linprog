%!TEX root =  autocontgrlp.tex
\section{Constraint sampling}
One way to reduce the number of constraints is to randomly choose a subset of them \cite{CS},
leading to the so-called reduced linear program (RLP). \todoc{How about the results of Calafiore and Campi (2005, Math. Prog.) and others reviewed in \cite{CS}? Any follow-up we missed?}
While the objective function of the RLP is the same as that of the ALP, 
the RLP has only $m \ll nd$ constraints sampled from the original ALP according to a given probability distribution:
\begin{align}\label{rlp}
\begin{split}
\min_{r\in \N}\, &c^\top \Phi r\\
\text{s.t.}\mb &(\Phi r)(s)\geq g_a(s)+\alpha\sum_{s'}p_a(s,s')(\Phi r)(s'), \mb (s,a) \in \I\,,
\end{split}
\end{align}
where $\I$ is the index set containing $m$ sampled state-action pairs and
the extra constraint set $\N\subset \R^k$ is introduced to guarantee a bounded feasible set (hence, bounded solution).
It is assumed that $\N$ is compact and $\tr\in \N$. \todoc{How can one arrange for this in general?}
The index set $\I$ naturally gives rise to the definition of an $nd\times m$ constraint sampling matrix $\M$:
\begin{definition}\label{csampmat}
Let $\I=\{(s_1,a_1),\ldots,(s_m,a_m)\}$ be the sampled state-action pairs and let $q_i \eqdef s_i+(a_i-1)\times n$, $i=1,\ldots,m$. Define the constraint sampling matrix $\M \in \R^{nd\times m}$ associated with the index set $\I$ by
\begin{align}
\M(i,j)
=
\begin{cases}
1, & \text{ if } q_i = j\,;\\
0, & \text{ otherwise}.
\end{cases}
\end{align}
\end{definition}
The RLP can then be represented in short as
\begin{align}\label{rlpshort}
\begin{split}
\min_{r\in \N}\, &c^\top \Phi r\\
\text{s.t.}\mb & \M^\top E \Phi r \geq \M^\top H \Phi r\,.
\end{split}
\end{align}
Note that apart from $r\in \N$ the feasible set of the RLP is a superset of the feasible set of the ALP.

The RLP based on constraint sampling has been found to perform well empirically in application domains ranging from controlled queuing networks (see section $6.2$ of \cite{ALP} and section $7$ of \cite{SALP}) to Tetris (see \cite{CST} and section $6$ of \cite{SALP}). However, the theoretical guarantees for the RLP are available only under a restricted setting \cite{CS}. We present the main result of \cite{CS} on constraint sampling, and to that end define the following quantities:
%\begin{definition}\label{sampdist}
Given a policy $u$, a Lyapunov function $\psi$ in the column span of $\Phi$
and a probability distribution $c$ over the states
we define probability distributions $\mu_{u,c}$, $\mu_{u,c,\psi}$ and constant $\theta_{u,c,\psi}$ as 
\begin{align}
\mu_{u,c}^\top& \eqdef (1-\alpha)c^\top (I-\alpha P_u)^{-1},\nn\\
\mu_{u,c,\psi}(s,a)&\eqdef (d \, \mu_u^\top \psi)^{-1} \, \mu_{u}(s) \psi(s),\nn\qquad s\in \S,a\in  \A\,,\\
\theta_{u,c,\psi} &\eqdef\mfrac{(1+\beta_{\psi})}{2} \, \mfrac{\mu_{u,c}^\top \,\psi}{\norm{J^*}_{1,c}}\, \sup_{r\in \N}||J^*-\Phi r||_{\mn}.
\end{align}
%\end{definition}
Note that in the definition of $\theta_{u,c,\psi}$ the \emph{supremum} 
of the error of approximating $J^*$ by the LFA over $\N$ appears.
De Farias and Van Roy show through an example how this term can still be reasonable controlled  \cite{CS}.
\todoc{If we avoid this, this would be great!}

We now recall Theorem~$3.1$ of \cite{CS} below. \todoc{Assumptions? }
\begin{theorem}\label{csresult}
Let $\epsilon$ and $\delta$ be scalars in $(0,1)$,
$\theta = \theta_{u^*,c,\psi}$ where $u^*$ is an optimal policy.
Let $\I$ be an index set obtained by sampling $m$ state-action pairs 
	independently according to the distribution $\mu_{u^*,c,\psi}$,
	\todoc{Their Lyapunov function definition is slightly different. Which one is the real one needed by this result?}
	where%
	\footnote{Note that the cardinality of $\I$ might be smaller than $m$.}
\begin{align}
m\geq \frac{16d\theta}{(1-\alpha)\epsilon}\left(k\ln\frac{48d\theta}{(1-\alpha)\epsilon}+\ln\frac{2}{\delta}\right).
\end{align}
Let $\tilde{r}_c$ be an optimal	solution of the ALP and let $\hat{r}_c$ be the solution of the corresponding RLP. 
Assume that $\tilde{r}_c\in \N$.
Then, with probability at least $1-\delta$, we have
\begin{align}
||J^*-\Phi \hat{r}_c||_{1,c}\leq ||J^*-\Phi \tilde{r}_c||_{1,c}+\epsilon||J^*||_{1,c}.
\end{align}
\end{theorem}
\noindent \textbf{Motivation for our work:}
As noted by the authors of \cite{CS}, this result is best viewed as a
	theoretical illustration of the potential of constraint sampling:
	sampling from $\mu_{u^*,c,\psi}$ will more often than not be infeasible in practice
	as this distribution is dependent on the optimal policy $u^*$.
However, as mentioned before, the RLP has performed reasonably well 
	even when the sampling distribution is not $\mu_{u^*,\psi}$. 
Thus there is a gap between the theoretical understanding of the RLP 
	and its practical efficacy which merits our attention. 
The gap also indicates that the RLP might be a special case 
	of a generalized constraint reduction scheme with provable performance guarantees. 
Understanding such a generalized method would result in an ALP based ADP technique 
	that would have theoretical performance guarantees while being practically useful. 
In particular, we wish to answer the following open questions.
\begin{itemize}
\item  As a natural generalization of the RLP, what happens if we define a generalized reduced linear program (GRLP) whose constraints are positive linear combinations of the original constraints of the ALP?
\item Unlike \cite{CS} which provides error bounds for a specific RLP formulated using an idealized sampling distribution, is it possible to provide error bounds for any GRLP (and hence any RLP)?
\end{itemize}
In this paper, we address both of the questions above.
