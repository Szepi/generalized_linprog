%!TEX root =  autocontgrlp.tex
\section{A worst-case bound}
The following lemmas relate the fixed point $\hv$ of $\hg$ to the solution $\hj$ of the GRLP in \eqref{grlp}.
\begin{lemma}\label{srw}
A vector
$\hat{r} \in \R^k$ is a solution to GRLP \eqref{grlp} iff it solves the following program:
\begin{align}\label{grlpeqprog}
\begin{split}
\min_{r\in \R^k}\, &||\Phi r-\hv||_{1,c}\\
\text{s.t.}\mb &\, W^\top E \Phi r\geq W^\top H \Phi r\,, \qquad r\in \N\,.
\end{split}
\end{align}
\end{lemma}
\begin{proof}
We know from Lemma~\ref{relation2} that $\hj = \Phi \hr \geq\hv$, and thus
the solutions to \eqref{grlp} do not change if we add the constraint $\Phi r \ge \hv$.
Now, under this constraint, minimizing $c^\top \Phi r$ is the same as
 minimizing 
\begin{align*}
||\Phi r-\hv||_{1,c}=\sum_{i=1}^n c(i) |(\Phi r)(i)-\hv(i)|=c^\top \Phi r-c^\top \hv\,.
\end{align*} 
\end{proof}
\begin{theorem}\label{mt2}
%Let $\hv$ be the solution to the iterative scheme in \eqref{apvi} and let $\hj=\Phi \hr$ be the solution to the GRLP. Let $\bj$ be the best possible approximation to $J^*$ as in Definition~\ref{bestproj}, and $||\Gamma J^* -\hg J^*||_\infty$ be the error due to ALUB projection and let $r^*\eqdef\underset{r\in \R^k}{\arg\min}||J^*-\Phi r||_\infty$, then
It holds that
\begin{align}
||\hj-\hv||_{1,c}\leq\frac{4||J^*-\Phi r^*||_\infty+||\Gamma J^*-\hg J^*||_\infty}{1-\alpha}.
\end{align}
\end{theorem}
\begin{proof}
Let $\gamma=||J^*-\Phi r^*||_\infty$. Then, since $T$ is an $\alpha$-contraction in the $\max$-norm,
$||J^*-T\Phi r^*||_\infty=||TJ^*-T\Phi r^*||_\infty\leq\alpha\gamma$ and hence by the triangle inequality,
\begin{align}\label{eq:tphir}
||T\Phi r^*-\Phi r^*||_\infty&\leq(1+\alpha)\gamma\,.
\end{align}
Let $r'= r^*+t e_1$ with $t>0$ to be chosen later.
Then, from \cref{one},
$T\Phi r' = T(\Phi r^* + t \one) = T \Phi r^* + \alpha t \one \le \Phi r^* + (\alpha t+ (1+\alpha) \gamma) \one$,
 where the inequality follows from~\eqref{eq:tphir}.
 Choosing $t$ so that $t = \alpha t+ (1+\alpha) \gamma$, or $t = \frac{(1+\alpha)\gamma}{1-\alpha}$, we get
 that $T\Phi r' \le \Phi r'$. Hence, $r'$ is feasible to the ALP \eqref{alp} and 
since, thanks to \cref{wassump}, 
any feasible solution to \eqref{alp} is also a feasible solution to the GRLP  \eqref{grlp},
$r'$ is also feasible to \eqref{grlp}. Thus, 
\begin{align}
||\hj-\hv||_{1,c}&\leq||\Phi r'-\hv||_{1,c}\nn &(\text{thanks to Lemma~\ref{srw}}) \\ %\vspace{10pt}
&\leq||\Phi r'-\hv||_\infty\mb & (\text{since $c$ is a distribution})\nn\\
&\leq||\Phi r'-J^*||_\infty+||J^*-\hv||_\infty\,. & (\text{triangle inequality}) \label{mt2eq0}
\end{align}
Now, by another triangle inequality,
\begin{align}
||\Phi r'-J^*||_\infty&\leq ||\Phi r^* -J^*||_\infty+||\Phi r'-\Phi r^*||_\infty
\leq \gamma+\frac{(1+\alpha)\gamma}{1-\alpha}=\frac{2\gamma}{1-\alpha}.
\label{mt2eq1}
\end{align}
The result follows by combining this bound with \eqref{mt2eq0} and \cref{cmt1}.
\end{proof}

\begin{corollary}[Prediction error bound]
\label{cmt2}
%Let $\hj$, $\hv$, $r^*$ and $J^*$ be as in Theorem~\ref{mt2}, then \todoc{Where is the effect of $\chi$?}
We have
\begin{align}\label{finalbnd}
||J^*-\hj||_{1,c}\leq\frac{6 ||J^*-\Phi r^*||_\infty+2||\Gamma J^*-\hg J^*||_\infty}{1-\alpha}\,.
\end{align}
\end{corollary}
\begin{proof}
We have
$
||J^*-\hj||_{1,c}\leq||J^*-\hv||_{1,c}+||\hv-\hj||_{1,c}
\leq||J^*-\hv||_\infty+||\hv-\hj||_{1,c}
$.
The result now follows from \cref{cmt1} and \cref{mt2}.
\end{proof}
When no constraints are removed, $\Gamma = \hg$. In this case, \eqref{finalbnd} can be compared to the bound in \cref{restateval} instantiated with $\psi = \one$, in which case the right-hand side of \eqref{finalbnd}
becomes $\frac{2\norm{J^* - \Phi r^*}_{\infty}}{1-\alpha}$, i.e., the bound  \eqref{finalbnd} is a off by a factor three 
only from that in \cref{restateval}.
We view this as a small price paid to be paid for the increased generality of \eqref{finalbnd}.
In terms of the price of removing constraints,
the result shows that there is effectively no price to be paid for removing constraints as long as
those constraints that are binding in the definition of $\Gamma J^*(i)$, $1\le i \le n$ are kept: 
a rather intuitive result.
\todoc{I am still trying to understand whether it is good or bad that the result is independent of $\N$.
}

Just like in the case of RLPs, it is possible to tighten the bounds by using weighted $\max$-norms. 
This is the subject of the next section.



