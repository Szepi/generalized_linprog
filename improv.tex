%!TEX root =  autocontgrlp.tex
\section{Bounds using weighted norms}\label{sec:improv}
In this section, we present error bounds by making use of Lyapunov functions and weighted norms.
For the purpose of this section, we define $r^*$ to be 
\begin{align*}
r^*\eqdef\argmin_{r\in R^k}||J^*-\Phi r||_{\mn}\,.
\end{align*}
where $\psi$ is a Lyapunov function as in \cref{lyap}.
We also replace \cref{ass:n} \eqref{ass:n1} by the following assumption,
which is assumed to hold in the rest of this section:
\begin{assumption}\label{ass:n4}
The set $\N$ is such that $\N = \N + t r_0$ for any $t\in \R$, where $r_0\in \R^k$ 
\todoc{Is not this a little problematic? In particular, this seems to contradict \cref{ass:n} \eqref{ass:n2}.
How can we guarantee that $\Gamma$, $\hg$ are well-defined?
}
is the vector such that $\Phi r_0 = \psi$.
\end{assumption}
\begin{lemma}\label{bestbndmn}
We have
\begin{align}
||J^*-\Gamma J^*||_{\mn}\leq 2||J^*-\Phi r^*||_{\mn}.
\end{align}
\end{lemma}
\begin{proof}
The proof parallels that of \cref{bestbnd}.
%By \cref{lyap}, there exists $r_0 \in \R^k$ such that $\psi = \Phi r_0$.
Define $\eps = \norm{J^* - \Phi r^*}_{\mn}$ so that
$|J^*(i) - (\Phi r^*)(i) | \le \eps \psi (i)$, $1 \le i \le n$.
Then, $\Phi( r^* + \eps r_0 ) = \Phi r^* + \eps \psi \ge J^* = T J^*$.
From this point the proof follows the same steps as those in \cref{bestbnd}
with the obvious modifications.
%The result follows from the definition of $\Gamma$ in \eqref{gamdef}, Assumption~\ref{lyap} and the fact that $\Phi r^*+||J^*-\Phi r^*||_{\mn} \psi\geq TJ^*$.
\end{proof}

Since most of our analysis in sections~\ref{sec:lubp} and ~\ref{sec:alubp} depended on showing that $\Gamma$ is a contraction map in the $L_\infty$ norm, we first show that $\Gamma$ is also a contraction map in the weighted $L_\infty$ norm.
The generalization of  \cref{maxnorm} for the weighted $L_{\infty}$ norm is as follows:
\begin{lemma}\label{maxnormmn}
Assume that $\Gamma: \R^n \to \R^n$ is monotone and 
that there exists some $\beta\in [0,1)$ such that for any $J\in \R^n$ and $t>0$,
\begin{align}
\label{eq:shiftmn}
\Gamma( J + t \psi) \le \Gamma J + \beta t \psi\,.
\end{align} 
for any $J\in \R^n$ and $t\ge 0$.
Then $\Gamma$ is a $\norm{\cdot}_{\mn}$ contraction with factor $\beta$.
\end{lemma}
\begin{proof}
The proof mimics that of  \cref{maxnorm}:
First, we show that for any $t\ge 0$,  $J\in \R^n$,
$\Gamma( J - t \psi) \ge \Gamma J - \beta t \psi$ also holds.
To see this define $J' = J-t\psi$. Then, $J = J'+t\psi$, hence $\Gamma J \le \Gamma J' + \beta t \psi$. Reordering this inequality gives the result.
Let $\eps = \norm{J_1 - J_2}_{\mn}$, where $J_1,J_2\in \R^n$ are arbitrary.
Then $J_2 - \eps \psi \le J_1 \le J_2 + \eps \psi$. 
By the monotonicity of $\Gamma$,
$\Gamma(J_2 - \eps \psi) \le \Gamma J_1 \le \Gamma(J_2 + \eps \psi)$. 
Using~\eqref{eq:shiftmn}, we get 
$\Gamma J_2 - \beta \eps \psi \le \Gamma J_1 \le \Gamma J_2 + \beta \eps \psi$, i.e., $-\beta \eps \psi \le \Gamma J_1 - \Gamma J_2 \le \beta \eps \psi$, from which the result follows.
\end{proof}
Let us note in passing that from this result and~\eqref{eq:linpsi} it immediately follows that $T$ is an $\norm{\cdot}_{\mn}$-contraction with factor $\beta_{\psi}$.
Returning to $\Gamma$,
since we already now that $\Gamma$ is monotone (cf. \cref{gmonotone}), it remains to see that it satisfies \eqref{eq:shiftmn}:
\begin{lemma}\label{gshiftmn}
The operator $\Gamma$ satisfies \eqref{eq:shiftmn} with $\beta = \beta_\psi$.
%For any $J\in \R^n$, $t\in \R_+$, $\Gamma (J+ t \psi ) \le \Gamma J + \beta_{\psi} t \psi$.
\end{lemma}
\begin{proof}
The proof of the first part parallels that of \cref{gshift}:
%Consider the $i^{th}$ linear programs associated with $\Gamma J$ and $\Gamma (J+ t\one)$. 
By definition, for $1\le i \le n$,
$(\Gamma (J+t\psi) )(i) = \min\{ e_i^\top \Phi r \,:\, \Phi r \ge T(J+t\psi), r\in \N \}$.
By \cref{eq:psilin}, as $t>0$, $T(J+t\psi) \le TJ + t \beta \psi$. Thus,
$(\Gamma (J+t\psi) )(i) \le 
 \min\{ e_i^\top \Phi r \,:\, \Phi r \ge TJ+t\beta \psi, r\in \N \}$.
Now, using Lemma~\ref{lpsol} with $A=\Phi$, $b=TJ$, $c=e_i$,
 $b_0=\beta t \psi$ 
and $x_0=t \beta r_0$, the statement follows
as thanks to \cref{lyap}, $A x_0 = b_0$ and thanks to
\cref{ass:n4}, $\N = \N + \alpha t r_0$.
%The result follows in a similar manner as the proofs for Lemmas~\ref{gshift} and ~\ref{tgshift} by using the result in Lemma~\ref{lpsol}.
\end{proof}
From this result and \cref{maxnormmn}, we immediately get that $\Gamma$ is a contraction in $\norm{\cdot}_{\mn}$:
\begin{theorem}\label{gmaxcontramn}
The operator $\Gamma  \colon \R^n\ra \R^n$ is a contraction operator in $\norm{\cdot}_{\mn}$ with factor $\beta_{\psi}$.
\end{theorem}
\if0
\begin{proof}
Given $J_1,J_2\in \R^n$ let $\epsilon=||J_1-J_2||_{\mn}$. Thus
\begin{align}\label{ineq}
J_2-\epsilon\psi\leq J_1\leq J_2+\epsilon \psi.
\end{align}
From Lemmas~\ref{gmonotone} and ~\ref{gshiftmn}, we can write
\begin{align}\label{ineq}
\Gamma J_2-\beta_{\psi} \epsilon\psi\leq \Gamma J_1\leq \Gamma J_2+\beta_{\psi} \epsilon\psi.
\end{align}
Thus
\begin{align}
||\Gamma J_1-\Gamma J_2||_{\mn}\leq \beta_{\psi} ||J_1-J_2||_{\mn}.
\end{align}
\end{proof}
\fi
In a similar way, one can also show that $\hg$ is also a contraction:
\begin{theorem}\label{hgmaxcontramn}
The operator $\hg:\R^n\to\R^n$  is a contraction operator in $\norm{\cdot}_{\mn}$ with factor $\beta_{\psi}$.
%$\hg$ is also a contraction map in the weighted $L_\infty$ norm.
\end{theorem}
\begin{proof}
We already know that $\hg$ is monotone. That $\hg$ satisfies~\cref{eq:shiftmn}
with $\beta = \beta_{\psi}$ follows similarly to the argument used in  \cref{tgshift}
with modifications similar to those introduced in the proof of \cref{gshiftmn}.
Then, \cref{gmaxcontramn} gives the desired result.
\end{proof}
\begin{lemma}\label{cmt1mn}
Let $\hv$, $\bj$ be as in Theorem~\ref{mt1} and let $r^*\eqdef\argmin_{r\in \R^k}||J^*-\Phi r||_{\mn}$ then
\begin{align}
||J^*-\hv||_{\mn}\leq \frac{2||J^*-\Phi r^*||_{\mn}+||\Gamma J^*-\hg J^*||_{\mn}}{1-\beta_{\psi}}.
\end{align}
\end{lemma}
\begin{proof}
The proof follows from \cref{gmaxcontramn}, \cref{hgmaxcontramn} and by replacing the $||\cdot||_\infty$ norm by $||\cdot||_{\mn}$ in the arguments presented in 
Sections~\ref{sec:lubp} and ~\ref{sec:alubp} leading to \cref{cmt1}.
\end{proof}

We now recall Lemma~$5$ from Section 4.2 of \cite{ALP}. 
For this result, recall that $r_\in \R^k$ is the vector such that $\psi = \Phi r_0$ by assumption.
\begin{lemma}\label{restate}
%Let $\psi$ be a Lyapunov function that belongs to the column span of $\Phi$ ,
For  $r \in \R^k$ arbitrary vector, let $r'$ be
\begin{align}
 r'= r+||J^*-\Phi r||_{\mn} \left(\frac{1+\beta_{\psi}}{1-\beta_{\psi}}\right)\, r_0.
\end{align}
Then $r'$ is feasible for the ALP in \eqref{alp}. 
\end{lemma}
Recall that $\hv$ is the fixed point of $\hg$ and $\hj=\Phi \hr$ is the solution to the GRLP
\eqref{grlp}. 
\begin{theorem}\label{mt2mn}
We have
\begin{align}
||\hj-\hv||_{1,c}&\leq \frac{c^\top \psi}{1-\beta_\psi}(4||J^*-\Phi r^*||_{\mn}
%\nn\\&
+||\Gamma J^*-\hg J^*||_{\mn}).
\end{align}
\end{theorem}
\begin{proof}
Let $\gamma=||J^*-\Phi r^*||_{\mn}$. 
Then, by choosing $r'$ as in Lemma~\ref{restate} we have
\begin{align*}
||\Phi r'-J^*||_{\mn}&\leq ||\Phi r^*-J^*||_{\mn}+||\Phi r'-\Phi r^*||_{\mn}\nn
=\gamma+\frac{1+\beta_\psi}{1-\beta_\psi}\gamma\nn
=\frac{2}{1-\beta_\psi}\gamma.\nn
\end{align*}
Since $r'$ is also feasible for the GRLP in \eqref{grlp} we have
\begin{align}
||\hj-\hv||_{1,c}&\leq||\Phi r'-\hv||_{1,c}\nn
=\sum_{s\in S}c(s)\psi(s)\frac{|\Phi r'(s)-\hv(s)|}{\psi(s)}\nn\\
&\leq c^\top \psi ||\Phi r'-\hv||_{\mn}\nn
\leq c^\top \psi (||\Phi r'-J^*||_{\mn}+||J^*-\hv||_{\mn}).
\end{align}
The result follows from \cref{cmt1mn}.
%\textbf{Main Result~$1$: Prediction Error bound in weighted $L_\infty$-norm}
\end{proof}
\begin{theorem}[Prediction error bound in $\norm{\cdot}_{\mn}$]
\label{cmt2mn}
%Let $\hj$, $\hv$, $r^*$ and $J^*$ be as in Theorem~\ref{mt2mn}, then
It holds that
\begin{align}\label{finalbndmn}
||J^*-\hj||_{1,c}&\leq\frac{c^\top\psi}{1-\beta_\psi}(6 ||J^*-\Phi r^*||_{\mn}
%\nn\\&
+2||\Gamma J^*-\hg J^*||_{\mn}).
\end{align}
\end{theorem}
\begin{proof}
We have
\begin{align}
||J^*-\hj||_{1,c}
&\leq||J^*-\hv||_{1,c}+||\hv-\hj||_{1,c}\nn
\leq c^\top \psi ||J^*-\hv||_{\mn}+||\hv-\hj||_{1,c}.\nn
\end{align}
The result now follows from Lemma~\ref{cmt1mn} and Theorem~\ref{mt2mn}.
\end{proof}

