%!TEX root =  autocontgrlp.tex
\section{Approximate Linear Programming}
The LP formulation in \eqref{mdplp} can be represented in short as,
\begin{align}
\begin{split}
\label{mdplpshort}
\min_{J\in \R^n}\, &c^\top J\\
\text{s.t.}\mb &J\geq T J,
\end{split}
\end{align}
or 
%\text{ or }\nn\\
\begin{align}
\begin{split}
\label{shortalter}
\min_{J\in \R^n} \,&c^\top J\\
\text{s.t.}\mb &EJ\geq H J,
\end{split}
\end{align}
where $J\geq TJ$ is a shorthand for the $nd$ constraints in \eqref{mdplp} and $E$ is an $nd\times n$ matrix given by $E=[I,\ldots,I]^\top$, i.e., $E$ is obtained by stacking $d$ identical $n\times n$ identity matrices one over the other. Note that \eqref{mdplpshort} and \eqref{shortalter} are identical programs and differ only in notation. We use notation of type \eqref{mdplpshort} whenever we prefer brevity and we use notation \eqref{shortalter} in some definitions and proofs for the sake of clarity.
Recall that the vector $c\in \R^n$ must have positive entries to guarantee that the unique solution to the above LP is the optimal value function. By convention, we will also assume that $c$ is also normalized:
\begin{assumption}\label{probdist}
$c=(c(i),i=1,\ldots,n)\in \R^n$ is a positive probability distribution, i.e., $c(i)>0$ and $\sum_{i=1}^n c(i)=1$.
\todoc{I am pretty sure the positivity is needed.}
\end{assumption}

The approximate linear program (ALP) is obtained by making use of LFA in the LP, i.e., by adding the extra constraint $J=\Phi r$ in \eqref{mdplpshort} with $\Phi \in \R^{n\times k}$ and the new variables $r\in \R^k$. 
By substitution, this leads to
\begin{align}\label{alp}
\begin{split}
\min_{r\in \R^k}\, &c^\top \Phi r\\
\text{s.t.}\mb &\Phi r\geq T \Phi r.
\end{split}
\end{align}
Unless specified otherwise we use $\tr$ to denote any solution\todoc{instead of saying the solution} to the ALP and $\tj=\Phi \tr$ to denote the corresponding approximate value function. 
The extra constraint added that $J$ lies in the (column) span of $\Phi$ could lead to an infeasible LP. An easy way to prevent this is if the vector $\one$ whose components are all equal to one is in the span of $\Phi$.
Indeed, by \cref{shift}, for any $t\in \R$, $T (t\one ) =  \alpha t \one+T \zero \le \alpha t \one + \max_{s,a} g_a(s)$, hence $t\one \ge \alpha t \one+T\zero $ happens as soon as
$t\ge \frac{1}{1-\alpha}\max_{s,a} g_a(s)$. 
This leads to the following assumption:
\begin{assumption}\label{one}
The first column of the feature matrix $\Phi$ (i.e., $\phi_1$) is $\one \in \R^n$. 
In other words, the constant function is part of the basis.
\end{assumption}

We will now recall two results due to de Farias and Van Roy 
	that bound the error due to the introduction of the extra constraint in the ALP.
For this, we will need the definition of weighted $L_1$ and $L_\infty$ norms.
For any $c,\rho:S \to \R$ positive valued functions
the weighted $L_1$-norms $\norm{\cdot}_{1,c}$ 
and 
the weighted $L_\infty$-norms  $\norm{\cdot}_{\infty,\rho}$ are defined as
\begin{align*}
||J||_{1,c}=\sum_{s \in S} c(s)|J(s)|\,, \qquad
||J||_{\infty,\rho}=\max_{s \in S} \frac{|J(s)|}{\rho(s)}\,, \qquad J:S\to \R\,.
\end{align*}

We let $\R_+$ denote the set of nonnegative reals. \todoc{Move up?}
For any function $\chi\colon S\ra \R_+$, define the discounted maximal inflation of $\chi$ due to $P = (p_a)_{a\in A}$ as
\begin{align*}
\beta_{\chi}=\max_{s \in S} \frac{\underset{a \in A}{\max}\big(\alpha\sum_{s'}p_a(s,s')\chi(s')\big)}{\chi(s)}\,.
\end{align*}
\begin{definition}
A function $\chi:S\to\R_+$ is said to be a \emph{Lyapunov} function for $P = (p_a)_{a\in A}$ 
	if $\beta_{\chi}<1$.
\end{definition}
The error bound will be given in terms of a Lyapunov function $\psi$ for $P$ that is in the column span of $\Phi$:
\begin{assumption}\label{lyap}
$\psi\colon S \ra \R_+$ is a Lyapunov function for $P$
and is present in the column span of the feature matrix $\Phi$.
\end{assumption}
It is straightforward to check that the vector $\one$ when viewed as an $S \to \R_+$ function 
is a Lyapunov function. 
Further, by \cref{one}, $\one$ is trivially present in the column span of $\Phi$, 
hence, \cref{lyap} is not limiting.
In what follows we will always assume that Assumptions \ref{probdist}--\ref{lyap} hold. 

%In Definition~\ref{modnorm}, the use of the weighting function $\rho$ allows us to measure the error taking into account the relative importance of the various states. A higher value of $\rho(s)$ means that the state $s$ is less important and vice-versa.\\
With this, we can recall the said result.  %below  which bounds the error in the approximate value function.
\begin{theorem}[Theorem~$4.2$ of \cite{ALP}]
\label{restateval}
Let $\tr$ be the solution to the ALP in \eqref{alp}, $\tilde{J}_c=\Phi \tilde{r}_c$.
% $\psi$ be as in \cref{lyap} 
%and $c$ be a distribution as in \eqref{probdist}. 
Then, it holds that
\begin{align*}
||J^*-\tj||_{1,c}\leq \frac{2c^\top \psi}{1-\beta_{\psi}}\min_{r}||J^*-\Phi r||_{\mn}\,.
\end{align*}
\end{theorem}
The next result  the loss in performance of a policy that is greedy w.r.t. $J_{\tu}$:
\begin{theorem}[Theorem~$3.1$ of \cite{ALP}]
\label{restatepol}
Let $\tilde{u}$ be the greedy policy with respect to the solution $\tj$ of the ALP. Then,
\begin{align*}
||J^*-J_{\tu}||_{1,c}\leq \frac{1}{1-\alpha}||J^*-\tj||_{1,c'}\,,
\end{align*}
where $c'=(1-\alpha)c^\top(I-\alpha P_{\tu})^{-1}$.
\end{theorem}
Theorems~\ref{restateval} and ~\ref{restatepol} together imply that the ALP addresses both the control and prediction problems. Please refer to \cite{ALP} for a more detailed treatment of the ALP.\\
Note that the ALP is a linear program in $k$ ($\ll n$) variables as opposed to the LP in \eqref{mdplpshort} which has $n$ variables. Nevertheless, the ALP has $nd$ constraints (same as the LP) which is an issue when $n$ is large and calls for constraint approximation/reduction techniques.
\todoc{Column generation, Dantzig-Wolf decomposition?}
