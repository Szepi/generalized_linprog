%!TEX root =  autocontgrlp.tex
\section{Error Analysis}\label{sec:improv}
\subsection{Properties of the Bellman Operator}
We now state without proof the most important properties of the Bellman operator(s).
The proofs are immediate from the definitions, but can also be found in \cite{BertB}.
\begin{comment}
First, we introduce some extra notation:
For $J_1,J_2\in \R^n$, we write $J_1\le J_2$ if $J_1(s)\le J_2(s)$ holds for all $s\in S$.
We use $\one \in \R^n$ to denote a vector with all entries $1$.
The maximum norm $\norm{\cdot}_{\infty}$ is defined by $ \norm{v}_{\infty} = \max_{s\in S} |v(s)|$.
\end{comment}
\begin{lemma}\label{tprop} The following hold:
\begin{enumerate}[(i)]
\item \label{monotone}$T$ is a monotone map, i.e., given $J_1,J_2 \in \R^n$ such that $J_1\leq J_2$, we have $T J_1\leq T J_2$.
\item \label{shift}
Given $J\in \R^n$ and $t \in \R$, we have $T(J+t\one)=TJ+\alpha t\one$.
\item \label{maxnorm}
If $T: \R^n \to \R^n$ is any operator that is monotonous and satisfies~\eqref{shift} then
$T$ is a $\max$-norm contraction operator with contraction factor $\alpha \in (0,1)$, i.e., given $J_1, J_2 \in \R^n$,
$
\norm{TJ_1-TJ_2}_\infty\leq \alpha \norm{J_1-J_2}_\infty.
$
\item \label{uniquesol}
$J^*$ is a unique fixed point of $T$, i.e., $J^*=TJ^*$.
\end{enumerate}
\end{lemma}
\begin{corollary}
If $J\in \R^n$ is such that $J\geq TJ$ then $J\geq TJ^2\geq \ldots \geq J^*$.
\end{corollary}
Though \cref{tprop} are stated for the Bellman operator $T$, the results also hold for $H$ as well.\par
We now present the analysis to derive the improved bounds where the idea is to show that the novel projection operators ($\Gamma/\hg$) are contraction maps. To this end, we go through steps similar to \Cref{tprop}-\eqref{monotone},~\eqref{shift} and ~\eqref{maxnorm}. Much the results that ensue are based on `Lyapunov' analysis where the idea is to replace the constant function $\one$ by a certain Lyapunov function (see \Cref{def:lyap}) and the $\norm{\cdot}_{\infty}$ by a weighted max-norm (see \Cref{notations}-\eqref{norms}).
\subsection{Analysis using Lyapunov Functions}
\begin{definition}\label{def:lyap}
Let $c,\rho,\chi:S \to \R_+$ be positive-valued functions. Then for $J\in \R^n$, $a\in A$ and $s\in S$, define
the discounted maximal inflation of $\chi$ due to $P = (p_a)_{a\in A}$ as $\beta_{\chi}=\max_{s \in S} \frac{\underset{a \in A}{\max}\big(\alpha\sum_{s'}p_a(s,s')\chi(s')\big)}{\chi(s)}$.
The function $\chi:S\to\R_+$ is a \emph{Lyapunov} function for $P = (p_a)_{a\in A}$ if $\beta_{\chi}<1$.
\end{definition}
\begin{assumption}\label{grlpassmp}
\begin{enumerate}[(i)]
%\item \label{one} The first column of the feature matrix $\Phi$ (i.e., $\phi_1$) is $\one \in \R^n$.
\item \label{lyap} $\psi\colon S \ra \R_+$ is a Lyapunov function for $P$
and is present in the column span of the feature matrix $\Phi$: For some $r_0\in \R^k$, $\Phi r_0 = \psi$.
\item \label{ass:n4} The set $\N'$ is such that $\N' = \N + t r_0$ for any $t\in \R$, where $r_0\in \R^k$ such that $\Phi r_0 = \psi$.
\item \label{wassmp} $W \in \R^{nd\times m}_+$ is a matrix with all positive entries.
\end{enumerate}
\end{assumption}
The authors of \cite{ALP} express the error bounds in terms of $\frac{1}{1-\beta_{\psi}}$.  A smaller $\beta_{\psi}$ loosely implies `stability' of the the underlying MDP,  with smaller values representative of higher stability. Prior works in ALP literature \cite{ALP,SALP,CS} make use of Lyapunov functions based analysis to obtain error bounds that exploit the structure of the underlying MDP. In particular, the prior bounds suggests that ALP is likely to generate good approximations when the underlying MDP is stable. We also adopt a similar approach by stating our results using Lyapunov function based arguments.

We note in passing that if \cref{grlpassmp}-\eqref{lyap} holds, it follows that for any $J\in \R^n$ and $t>0$,
\begin{align}\label{eq:psilin}
\begin{split}
T(J+ t \psi ) \le TJ + \beta_{\psi}\,t\,  \psi,\\
H(J+ t \psi ) \le HJ + E \beta_{\psi}\,t\,  \psi\,.
\end{split}
\end{align}
%\begin{figure}[h!]
\centering
%\resizebox{columnwidth}{}{
\begin{tikzpicture}[domain=0:7.7,scale=0.7,font=\small,axis/.style={very thick, ->, >=stealth'}]
\draw [line,thick,-] (0,-1)--(0,4.7);
\draw [line,thick,-] (0,4.7)--(12.5,4.7);
\draw [line,thick,-] (12.5,4.7)--(12.5,-1);
\draw [line,thick,-] (0,-1)--(12.5,-1);
\node[](one) at (1.5,0.5) {$\tj$};
\node[](four) at (2.5,0.5) {$\geq$};
\node[](two) at (5,0.5) {$\tv$};
\node[](three) at (10,0.5) {$J^*$};
\node[](five) at (7.5,0.5) {$\geq$};
\node[](six) at (10,3.7) {$\hj$};
\node[](seven) at (5,3.7) {$\hv$};
\node[](eight) at (7.5,3.7) {$\leq$};
\node[](nine) at(1.5,0){\text{Solution }};
\node[](twenty) at(1.5,-0.5	){\text{to ALP }};
\node[](ten) at(5,0){\text{Fixed point of }$\Gamma$};
\node[](eleven) at(10,0){\text{Optimal Value Function }};
\node[](twelve)at (10,4.2){\text{Solution to GRLP}};
\node[](thirteen)at (5,4.2){\text{Fixed point of }$\hg$};
\draw [line,thick,<->] (5,3.2)--(9.8,0.9);
\draw [line,thick,<->] (5,3.2)--(9.8,3.2);
\draw [line,thick,<->] (9.8,3.2)--(9.8,0.9);
\node[](fourteen)at (6,2.2){$d_1$};
\node[](fifteen)at (6.5,2.9){$d_2$};
\node[](sixteen)at (11.2,2.2){$d\leq d_1+d_2$};
\node[](eighteen)at (11,1.6){\text{Corollary~\ref{cmt2}}};
\node[](seventeen)at (8,2.9){\text{(Theorem~\ref{mt2}})};
\node[](fourteen)at (6,1.8){\text{(Theorem~\ref{mt1})}};
\node[](fourteen)at (6,1.3){\text{(\& Corollary~\ref{cmt1})}};
%\node[rotate=-25](fourteen)at (7.5,1.5){\text{(Theorem~\ref{mt1})}};
%\node[draw, circle](two) at (5.5,1) {$s_{n+1}=s'$};
%\draw[->, =>latex](one) edge[bend left=42.5](two);
%\node [above=1.1cm]at (2.8,1.2) {${p_{a_n}(s,s')}$};
%\node [right] at(0,0){$g_{a_n}(s_n)$};
\end{tikzpicture}
%}
\caption{A schematic of the error analysis. Here $d=||J^*-\hj||_{1,c}.$}
\label{schematic}
\end{figure}


The next lemma is elementary, but will prove to be useful:
\begin{lemma}\label{lpsol}
Let $A\in \R^{u\times v}$, $b\in \R^u,d\in \R^v$ and $b_0=Ax_0$ for
some $x_0 \in \R^v$, $\N' \subset \R^v$ such that $\N' =x_0+ \N'$. Then
\begin{align}
&\min\{d^\top x:Ax\geq b+b_0, x\in \N'\} \nn\\
&=\min\{d^\top y:Ay \geq b, y \in \N' \}+d^\top x_0.
\end{align}
\end{lemma}
\begin{proof}
The claim follows by the change of variables $y := x-x_0$.
\end{proof}
\noindent 
%Note that under our assumptions on $\N$, $\hg$ is well-defined.
%\begin{align*}
%r^*\eqdef\argmin_{r\in R^k}\norm{J^*-\Phi r}_{\mn}\,.
%\end{align*}
%where $\psi$ is a Lyapunov function as in \cref{lyap}.
\begin{lemma}\label{bestbndmn}
Let $r^*\eqdef\underset{r\in \R^k}{\min}\parallel J^*-\Phi r\parallel_{\mn}$. Then,
\begin{align}
\norm{J^*-\Gamma J^*}_{\mn}\leq 2 \norm{J^*-\Phi r^*}_{\mn}.
\end{align}
\end{lemma}
\begin{proof}
Define $\eps = \norm{J^* - \Phi r^*}_{\mn}$ so that
$ J^*-\Phi r^* \le \eps \psi$ and $\Phi r^* - J^*\le \eps \psi$. Then, from \cref{grlpassmp}-\eqref{ass:n4} it follows that $\Phi( r^* + \eps r_0 ) = \Phi r^* + \eps \psi \ge J^* = T J^*$. From the definition of $\Gamma$ in \eqref{gamdef} and \cref{ubrem}, we know that $\Phi r^*+\eps \psi \geq \Gamma J^*\geq J^*$. The result follows by noting that $2\eps\psi\geq \Phi r^*+\eps\psi -J^*\geq \Gamma J^*-J^*\geq 0$.
\end{proof}
\begin{lemma}\label{tgmonotone}
For $J_1, J_2\in \R^n$ such that $J_1\leq J_2$, we have $\hg J_1\leq \hg J_2$.
\end{lemma}
\begin{proof}
Given $J\in \R^n$, let $\F_J\eqdef\{\,\Phi r\,: W^\top E \Phi r\geq W^\top HJ, r\in \N'\,\}\,$. Choose any $i\in \{1, \ldots, n\}$. Since $J_1\leq J_2$, from \Cref{tprop}-\eqref{monotone} and \cref{grlpassmp}-\eqref{wassmp} it follows that $W^\top H J_1\leq W^\top H J_2$. Hence, $\F_{J_2} \subset \F_{J_1}$ and thus $(\hg J_1)(i) \le (\hg J_2)(i)$.  Since $i$ was arbitrary, the result follows.
\end{proof}
\begin{lemma}\label{maxnormmn}
Assume that $\hg: \R^n \to \R^n$ is monotone and 
that there exists some $\beta\in [0,1)$ such that for any $J\in \R^n$ and $t>0$,
\begin{align}
\label{eq:shiftmn}
\hg( J + t \psi) \le \hg J + \beta t \psi.
\end{align} 
Then $\hg$ is a $\norm{\cdot}_{\mn}$ contraction with factor $\beta$.
\end{lemma}
\begin{proof}
First, we show that for any $t\ge 0$,  $J\in \R^n$,
$\hg( J - t \psi) \ge \hg J - \beta t \psi$ also holds.
To see this define $J' = J-t\psi$. Then, $J = J'+t\psi$, hence $\hg J \le \hg J' + \beta t \psi$. Reordering this inequality gives the result.
Let $\eps = \norm{J_1 - J_2}_{\mn}$, where $J_1,J_2\in \R^n$ are arbitrary.
Then $J_2 - \eps \psi \le J_1 \le J_2 + \eps \psi$. 
By the monotonicity of $\hg$,
$\hg(J_2 - \eps \psi) \le \hg J_1 \le \hg(J_2 + \eps \psi)$. 
Using~\eqref{eq:shiftmn}, we get 
$\hg J_2 - \beta \eps \psi \le \hg J_1 \le \hg J_2 + \beta \eps \psi$, i.e., $-\beta \eps \psi \le \hg J_1 - \hg J_2 \le \beta \eps \psi$, from which the result follows.
\end{proof}
\begin{corollary}\label{tmaxnormmn}
$T$ is a $\norm{\cdot}_{\mn}$-contraction with factor $\beta_{\psi}$.
\end{corollary}
%Returning to $\hg$, since we already now that $\hg$ is monotone (cf. \cref{gmonotone}), it remains to see that it satisfies \eqref{eq:shiftmn}:
\begin{lemma}\label{gshiftmn}
The operator $\hg$ satisfies \eqref{eq:shiftmn} with $\beta = \beta_\psi$.
%For any $J\in \R^n$, $t\in \R_+$, $\Gamma (J+ t \psi ) \le \Gamma J + \beta_{\psi} t \psi$.
\end{lemma}
\begin{proof}
By definition, for $1\le i \le n$, $(\hg (J+t\psi) )(i) = \min\{ e_i^\top \Phi r \,:\, W^\top E\Phi r \ge W^\top H(J+t\psi), r\in \N' \}$.
By \eqref{eq:psilin}, as $t>0$, $H(J+t\psi) \le HJ + t \beta_\psi \psi$ and hence $W^\top H(J+t\psi) \le W^\top (HJ + t \beta_\psi \psi)$. Thus,
$(\hg (J+t\psi) )(i) \le 
 \min\{ e_i^\top \Phi r \,:\, W^\top E\Phi r \ge W^\top HJ+t\beta_\psi \psi), r\in \N' \}$.
Now, using \Cref{lpsol} with $A=W^\top E \Phi$, $b=W^\top HJ$, $d=e_i\Phi$, $b_0=t\beta_\psi \psi$
and $x_0=t \beta_\psi r_0$, the statement follows since $A x_0 = b_0$ (from \cref{grlpassmp}-\eqref{lyap}) and $\N' = \N' + \alpha t r_0$ (from \cref{grlpassmp}-\eqref{ass:n4}).
\end{proof}
%From this result and \cref{maxnormmn}, we immediately get that $\Gamma$ is a contraction in $\norm{\cdot}_{\mn}$:
\begin{comment}
\begin{theorem}\label{gmaxcontramn}
The operator $\Gamma  \colon \R^n\ra \R^n$ is a contraction operator in $\norm{\cdot}_{\mn}$ with factor $\beta_{\psi}$.
\end{theorem}
\end{comment}
%In a similar way, one can also show that $\hg$ is also a contraction:
\begin{theorem}\label{hgmaxcontramn}
The operator $\hg:\R^n\to\R^n$  is a contraction operator in $\norm{\cdot}_{\mn}$ with factor $\beta_{\psi}$.
%$\hg$ is also a contraction map in the weighted $L_\infty$ norm.
\end{theorem}
\begin{proof}
Follows from \cref{maxnormmn,gshiftmn}.
\begin{comment}
We already know that $\hg$ is monotone. That $\hg$ satisfies~\cref{eq:shiftmn}
with $\beta = \beta_{\psi}$ follows similarly to the argument used in  \cref{tgshift}
with modifications similar to those introduced in the proof of \cref{gshiftmn}.
Then, \cref{gmaxcontramn} gives the desired result.
\end{comment}
\end{proof}
In what follows we denote by $\hv$ the unique fixed point of $\hg$, i.e., $\hv = \hg \hv\,$. 
We now show that vector $\hj$ dominates $\hv$:
\begin{lemma}\label{relation1}
The vectors $\hv,\hj$ obey $\hj\geq\hv$.
\end{lemma}
\begin{proof}
For $i\in \{1,\dots,n\}$, $c=e_i$ let $r_i$ be a solution to GRLP in \eqref{grlp}
and define $V_0\in \R^n$ by $V_0(i)=\underset{j=1,\ldots,n}{\min}(\Phi r_j)(i)$, $1\le i \le n$.

It suffices to show that $V_1\eqdef \hg V_0 \le V_0 \le \hj$ since then the desired result follows
by defining $V_{n+1} = \hg V_n$, $n\ge 1$, noting that by \cref{tgmonotone}, $V_{n+1}\le V_{n}$ and by  \cref{tmaxnormmn}, $V_n \to \hv$.

Since $(\Phi r_j)(i) \ge (\Phi r_i)(i)$ also holds for any $1\leq i,j\leq n$ we have $V_0(i)  = (\Phi r_i)(i)$. Also, since $\hj(i) \ge (\Phi r_i)(i),1\leq i \leq n$ it follows that $\hj\geq V_0$. 
Now,  fix any $i$. 
We need to show that $V_1(i)=(\hg V_0)(i) = (\Phi r'_{e_i,V_0})(i) \leq V_0(i)$. 
By the definition of $r'_{e_i,V_0}$ we know that $(\Phi r'_{e_i,V_0})(i) \le (\Phi r)(i)$
holds for any $r\in \N$ such that $W^\top E \Phi r \ge W^\top H V_0$. 
Now it suffices to show that $r_i$ satisfies $W^\top E \Phi r_i \ge W^\top H V_0$. 
By definition, $r_i$ satisfies $W^\top E \Phi r_i \ge W^\top H \Phi r_i$.
Hence, by the monotone property of $H$ and \cref{grlpassmp}-\eqref{wassmp} it is sufficient if $\Phi r_i \ge V_0$.
This however follows from the definition of $V_0$.
\end{proof}
\begin{lemma}\label{relation2}
The vectors $\hv,\tj$ obey $\tj\geq\hv$.
\end{lemma}
\begin{proof}
%The proof follows in a similar manner as the proof of \Cref{relation1}. To elaborate, 
Let $ r_1,  r_2,\ldots, r_n$ be solutions to ALP in \eqref{alp} (with an additional constraint that the solution be restricted to $\N$) for $c=e_1, e_2,\ldots,e_n$, respectively,
and define $V_0\in \R^n$ by $V_0(i)=\underset{j=1,\ldots,n}{\min}(\Phi r_j)(i)$, $1\le i \le n$. The rest of the proof 
follows in the same manner as the proof of \cref{relation1}.
\end{proof}
\begin{lemma}\label{srw}
A vector
$\hat{r} \in \R^k$ is a solution to GRLP \eqref{grlp} iff it solves the following program:
\begin{align}\label{grlpeqprog}
\begin{split}
\min_{r\in \N}\, &\norm{\Phi r-\hv}_{1,c}\,\mb
\text{s.t.}\mb \, W^\top E \Phi r\geq W^\top H \Phi r.
\end{split}
\end{align}
\end{lemma}
\begin{proof}
We know from \Cref{relation1} that $\hj = \Phi \hr \geq\hv$, and thus
the solutions to \eqref{grlp} do not change if we add the constraint $\Phi r \ge \hv$.
Now, under this constraint, minimizing $c^\top \Phi r$ is the same as minimizing 
\begin{align*}
\norm{\Phi r-\hv}_{1,c}=\sum_{i=1}^n c(i) |(\Phi r)(i)-\hv(i)|=c^\top \Phi r-c^\top \hv\,.
\end{align*} 
\end{proof}
\begin{lemma}\label{srwalp}
A vector
$\tr \in \R^k$ is a solution to ALP \eqref{alp} iff it solves the following program:
\begin{align}\label{grlpeqprog}
\begin{split}
\min_{r\in \R^k}\, &\norm{\Phi r-\hv}_{1,c}\,\mb
\text{s.t.}\mb \, \Phi r\geq  T \Phi r.
\end{split}
\end{align}
\end{lemma}
\begin{proof}
We know from \Cref{relation2} that $\tj = \Phi \tr \geq\hv$.
The rest of the argument follows in the same manner as the proof for \cref{srw}.
\end{proof}
\begin{lemma}\label{cmt1mn}
We have
\begin{align}
\norm{J^*-\hv}_{\mn}
& \leq \frac{1}{{1-\beta_{\psi}}}\big(2\norm{J^*-\Phi r^*}_{\mn}
\nn\\
& {}\qquad \qquad+\norm{\Gamma J^*-\hg J^*}_{\mn}\big).
\end{align}
\end{lemma}
\begin{proof}
Recall that $\hg \hv=\hv$. By the triangle inequality,
\begin{align*}
\norm{J^*-\hv}_{\mn}
& \leq \norm{J^*-\hg J^*}_{\mn}+\norm{\hg J^*-\hg \hv}_{\mn}\\
&\leq \norm{J^*-\hg J^*}_{\mn}+\beta_\psi \norm{ J^*- \hv}_{\mn},
\end{align*}
and so by reordering and with another triangle inequality,
\begin{align*}%\label{jhv}
\norm{J^*-\hv}_{\mn} \nn
&\le \frac{\norm{ J^*-\hg J^*}_{\mn}}{1-\beta_\psi}\nn\\
&\le \frac{\norm{ J^*-\Gamma J^*}_{\mn}+\norm{\Gamma J^* - \hg J^*}_{\mn}}{1-\beta_\psi}\,.
\end{align*}
The proof follows by combining this and \cref{bestbndmn}.
\end{proof}
We now recall Lemma~$5$ from Section 4.2 of \cite{ALP}. 
For this result, recall that $r_0 \in \R^k$ is the vector such that $\psi = \Phi r_0$.
\begin{lemma}\label{restate}
%Let $\psi$ be a Lyapunov function that belongs to the column span of $\Phi$ ,
For  $r \in \R^k$ arbitrary, let $r'$ be
\begin{align}
r'= r+\norm{J^*-\Phi r}_{\mn} \left(\frac{1+\beta_{\psi}}{1-\beta_{\psi}}\right)\, r_0.
\end{align}
Then $r'$ is feasible for ALP in \eqref{alp}.
\end{lemma}
Recall that $\hv$ is the fixed point of $\hg$ and $\hj=\Phi \hr$ is the solution to GRLP
\eqref{grlp}. 
\begin{theorem}\label{mt2mn}
We have
\begin{align}
\norm{\hj-\hv}_{1,c}&\leq \frac{c^\top \psi}{1-\beta_\psi}(4\norm{J^*-\Phi r^*}_{\mn}\nn
+\norm{\Gamma J^*-\hg J^*}_{\mn}).
\end{align}
\end{theorem}
\begin{proof}
Let $\gamma=\norm{J^*-\Phi r^*}_{\mn}$.
Then, by choosing $r'$ as in \Cref{restate} we have
\begin{align}
\norm{\Phi r'-J^*}_{\mn}\nn
&\leq \norm{\Phi r^*-J^*}_{\mn}+\norm{\Phi r'-\Phi r^*}_{\mn}\nn\\
&=\gamma+\frac{1+\beta_\psi}{1-\beta_\psi}\gamma
	=\frac{2}{1-\beta_\psi}\gamma.
\label{part}
\end{align}
%We know that $\tr \in \N$ by \cref{grlpassmp}-\eqref{nassmp} and 
Now, $r'$ is feasible for ALP in \eqref{alp} by \cref{restate}.
Then from \cref{srwalp} it follows that
%As a result, from \cref{srw} it follows that
\begin{align}
\norm{\hj-\hv}_{1,c}
&\leq\norm{\Phi \tr-\hv}_{1,c}\leq\norm{\Phi r'-\hv}_{1,c}\nn\\
&=\sum_{s\in S}c(s)\psi(s)\frac{|\Phi r'(s)-\hv(s)|}{\psi(s)}\nn\\
&\leq c^\top \psi \norm{\Phi r'-\hv}_{\mn}\nn\\
&\leq c^\top \psi (\norm{\Phi r'-J^*}_{\mn}+\norm{J^*-\hv}_{\mn}).\nn
\end{align}
The result follows from \cref{cmt1mn} and \eqref{part}.
%\textbf{Main Result~$1$: Prediction Error bound in weighted $L_\infty$-norm}
\end{proof}
\begin{theorem}[Prediction error bound]
\label{cmt2mn}
%Let $\hj$, $\hv$, $r^*$ and $J^*$ be as in Theorem~\ref{mt2mn}, then
It holds that
\begin{align}\label{finalbndmn}
\begin{split}
\norm{J^*-\hj}_{1,c}
&\leq\frac{c^\top\psi}{1-\beta_\psi}(6 \norm{J^*-\Phi r^*}_{\mn}\\
&+2\norm{\Gamma J^*-\hg J^*}_{\mn}).
\end{split}
\end{align}
\end{theorem}
\begin{proof}
We have
\begin{align}
\norm{J^*-\hj}_{1,c}\nn
&\leq\norm{J^*-\hv}_{1,c}+\norm{\hv-\hj}_{1,c}\nn\\
&\leq c^\top \psi \norm{J^*-\hv}_{\mn}+\norm{\hv-\hj}_{1,c}.\nn
\end{align}
The result now follows from \Cref{cmt1mn} and \Cref{mt2mn}.
\end{proof}
We now bound the performance of the greedy policy $\hu$.
\begin{theorem}[Control Error Bound]
\label{polthe}
Let $\hu$ be the greedy policy with respect to the solution $\hj$ of GRLP and $J_{\hu}$ be its value function.
% Let $r^*$ be as in Theorem~\ref{mt2mn}, then
Then,
\begin{align}\label{polthebnd}
\norm{J^* - J_{\hu}}_{1,c}
&\leq 2\left(\frac{c^\top \psi}{(1-\beta_{\psi})^2}\right)\, \big( 2\norm{J^*-\Phi r^*}_{\mn}
\nn\\&
+\norm{\Gamma J^*-\hg J^*}_{\mn}+\norm{\hj-\hg\hj}_{\mn}\big).
\end{align}
\end{theorem}
\begin{proof}
By the triangle inequality,
\begin{align*}
\norm{J^*-J_{\hu}}_{1,c}&\leq \norm{J^*-\hj}_{1,c}+\norm{J_{\hu}-\hj}_{1,c}\,.
\end{align*}
Let us now bound the second term on the right-hand side.
Since $\hu$ is greedy w.r.t. $\hj$, it holds that $T_{\hu} \hj = T \hj$.
Also, $T_{\hu} J_{\hu} = J_{\hu}$.
Hence, $J_{\hu} - \hj = T_{\hu} J_{\hu} - T_{\hu} \hj + T \hj - \hj
=\alpha P_{\hu} (J_{\hu}- \hj) + T\hj - \hj$.
Hence,
\begin{align}\label{polderv}
||J_{\hu}-\hj||_{1,c}\nn
&=||(I-\alpha P_{\hu})^{-1}(T\hj-\hj)||_{1,c}\nn\\
&\leq c^\top(I-\alpha P_{\hu})^{-1}|T\hj-\hj|\nn\\
&\leq c^\top (I-\alpha P_{\hu})^{-1} \,\psi\, \norm{T\hj-\hj}_{\mn}\nn\\
&\leq \frac{c^\top \psi}{1-\beta_{\psi}}\norm{T\hj-\hj}_{\mn}\nn\\
%&\leq \frac{c^\top \psi}{1-\beta_{\psi}}\norm{T\hj-TJ^* +J^*- \hj}_{\mn}\nn\\
&\leq \frac{c^\top \psi}{1-\beta_{\psi}}(\norm{T\hj-TJ^*}_{\mn} +\norm{J^*- \hj}_{\mn})\nn\\
&\leq \frac{c^\top \psi}{1-\beta_{\psi}}(1+\beta_{\psi})\norm{J^*- \hj}_{\mn},
\end{align}
where in the second inequality, we use Jensen's inequality and $|T\hj - \hj|$ stands for the 
vector whose $i$th component is $|(T\hj)(i) - \hj(i)|$. Further, the last inequality follows
since $T$ is a $\norm{\cdot}_{\mn}$ contraction with factor $\beta_{\psi}$ as noted earlier.
%componentwise  $(I-\alpha P_{\hu})^{-1}$ is a positive operator for $x=(x_1,\ldots,x_n)^\top\in \R^n$, $|x|=(|x_1|,\ldots,|x_n|)^\top\in \R^n$. 
Hence,
\begin{align}
&\norm{J^*-J_{\hu}}_{1,c}\nn\\
%&\leq \norm{J^*-\hj}_{1,c}+\norm{J_{\hu}-\hj}_{1,c}\nn\\
&\leq c^\top \psi \norm{J^*-\hj}_{\mn}+c^\top \psi\frac{1+\beta_\psi}{1-\beta_\psi}\norm{J^*- \hj}_{\mn}\nn\\
&=\frac{2c^\top \psi}{1-\beta_{\psi}}\norm{J^*- \hj}_{\mn}.
\end{align}
Now in a manner similar to \cref{cmt2mn} we have
\begin{align}
\norm{J^*- \hj}_{\mn}&\leq \norm{J^*- \hv}_{\mn}+\norm{\hv -\hj}_{\mn}\nn
\end{align}
The result now follows by substituting the bound on $\norm{J^*- \hv}_{\mn}$ from \cref{cmt1mn} and the fact that $\norm{\hv-\hj}_{\mn}\leq \frac{1}{1-\beta_{\psi}}\norm{\hj-\hg\hj}_{\mn}$.
\end{proof}
\begin{comment}
\begin{note}
By bounding $\etmn=\norm{\Gamma J^*-J^*+J^*-\hg J^*}_{\mn}\leq 2\norm{J^*-\Phi r^*}_{\mn}+\norm{J^*-\hg J^*}_{\mn}$
(the inequality follows from \cref{bestbndmn}), 
we can loosen the bounds in \cref{cmt2mn} and \cref{polthe} to
\begin{align}
\label{loose1}
\norm{J^*-\hj}_{1,c}&\leq\frac{c^\top\psi}{1-\beta_\psi}(10 \norm{J^*-\Phi r^*}_{\mn}
\nn\\&
+2\norm{J^*-\hg J^*}_{\mn}).\\
\label{loose2}
\norm{J^* - J_{\hu}}_{1,c}&\leq 2\left(\frac{c^\top \psi}{1-\beta_{\psi}}\right)^2 \,\big(10 \norm{J^*-\Phi r^*}_{\mn}
\nn\\&
+2\norm{J^*-\hg J^*}_{\mn}\big).
\end{align}
Here the term $||J^*-\hg J^*||$ in \eqref{loose1} and \eqref{loose2} captures the error due to the use of both $\Phi$ and $W$. Though, \eqref{loose1} and \eqref{loose2} might be loser bounds than \eqref{finalbndmn} and \eqref{polthebnd} respectively, the advantage of this bound is that it captures the error due to function approximation as well as constraint reduction in a direct manner.
\end{note}
\end{comment}
\begin{theorem}[Constraint Sampling]
Let $s\in S$ be a state whose constraint is selected by $W$ (i.e., for some $i$ and all $(s',a)\in S\times A$,
$W_{s'a,i}=\delta_{s=s'}$.
Then
$
|\Gamma J^*(s)-\hg J^*(s)|<|\Gamma J^*(s)-J^*(s)|.
$
\end{theorem}

\begin{proof}
Let $r_{e_s,J^*}$ and ${r}'_{e_s,J^*}$ be solutions to the linear programs in \eqref{lubplp} and \eqref{alubplp} respectively for $c=e_s$ and $J=J^*$. It is easy to note that $r_{e_s,J^*}$ is feasible for the linear program in \eqref{alubplp} for $c=e_s$ and $J^*$, and hence it follows that $(\Phi r_{e_s,J^*})(s)\geq (\Phi {r}'_{e_s,J^*})(s)$. However, since the constraints with respect to state $s$ have been chosen we know that $(\Phi {r}'_{e_s,J^*})(s)\geq J^*(s)$. The proof follows from noting that $(\Gamma J^*)(s)=(\Phi r_{e_s,J^*})(s)$ and $\hg J^*(s)=(\Phi {r}_{e_s,J^*})(s)$.
\end{proof}



