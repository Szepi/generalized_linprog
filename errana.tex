%!TEX root =  autocontgrlp.tex
\section{Error Analysis}\label{sec:errana}
We now define two projection operators which are central to our error analysis:
%The least upper bound (LUB) projection operator $\Gamma \colon \R^n \ra\R^n$ is defined below, see \eqref{gamdef}.
\begin{definition}\label{lubpop}
Given $J\in \R^n$ and the nonnegative valued vector $c\in \R^n_+$, define $r_{c,J}$ to be the solution to 
\begin{align}
\label{lubplp}
\begin{split}
 \underset{r\in \N'}{\min} &\,\, c^\top \Phi r\,\mb
 \text{s.t.} \mb \Phi r\geq  TJ.
 \end{split}
\end{align}
Then, for $J\in \R^n$, $\Gamma J$,
the \emph{ least upper bound projection} of $J$ is defined as 
\begin{align}\label{gamdef}
(\Gamma J)(i)\eqdef(\Phi r_{e_i,J})(i),\quad i=1,\ldots,n\,.
\end{align}
\end{definition}
\begin{remark}\label{ubrem}
To understand the meaning of $\Gamma$ (and $\hg$) define 
\begin{align}\label{ubclass}
\F_J\eqdef\{\,\Phi r\,:\,\Phi r\geq TJ, r\in \N\,\},
\end{align}
where $J\in \R^n$. 
Disregarding the constraint $r\in \N$, 
$\F_J$ contains vectors in the span of $\Phi$ that upper bound $TJ$. Further, since $(\Gamma J)(i) = \min\{ V(i) \,:\, V\in \F_J \}$, it also follows that $ V\ge \Gamma J $ holds for any $V\in \F_J$.
\end{remark}
%The approximate least upper bound (ALUB) projection operator $\hg \colon \R^n \ra \R^n$ is defined as below:
The definition of the second operator is as follows:
\begin{definition}\label{alubpop}
Given $J\in \R^n$ and the nonnegative valued vector $c\in \R^n_+$, define $r'_{c,J}$ to be the solution to 
\begin{align}\label{alubplp}
 \underset{r\in \N'}{\min}& \,\mb c^\top \Phi r\,\mb
 \text{s.t.} \,\,\, W^\top E \Phi r\geq W^\top HJ.
\end{align}
Then, the \emph{approximate least upper bound (ALUB) projection operator}
$\hg \colon \R^n \ra \R^n$ is defined as 
\begin{align}\label{tgamdef}
(\hg J)(i)\eqdef(\Phi r'_{e_i,J})(i), \mb i=1,\ldots,n\,, J\in \R^n\,.
\end{align}
\end{definition}
The next lemma is elementary, but will prove to be useful:
\begin{lemma}\label{lpsol}
Let $A\in \R^{u\times v}$, $b,c\in \R^u$ and $b_0=Ax_0$ for 
some $x_0 \in \R^v$, $\N' \subset \R^v$ such that $\N' =x_0+ \N'$. Then
\begin{align}
&\min\{c^\top Ax:Ax\geq b+b_0, x\in \N'\} \nn\\
&=\min\{c^\top A y:Ay \geq b, y \in \N' \}+c^\top b_0.
\end{align}
\end{lemma}
\begin{proof}
The claim follows by the change of variables $y := x-x_0$.
\end{proof}
\noindent 
%Note that under our assumptions on $\N$, $\hg$ is well-defined.
%\begin{align*}
%r^*\eqdef\argmin_{r\in R^k}||J^*-\Phi r||_{\mn}\,.
%\end{align*}
%where $\psi$ is a Lyapunov function as in \cref{lyap}.
\begin{lemma}\label{bestbndmn}
We have
\begin{align}
||J^*-\Gamma J^*||_{\mn}\leq 2||J^*-\Phi r^*||_{\mn}.
\end{align}
\end{lemma}
\begin{proof}
Define $\eps = \norm{J^* - \Phi r^*}_{\mn}$ so that
$ J^*-\Phi r^* \le \eps \psi$ and $\Phi r^* - J^*\le \eps \psi$. Then from \cref{grlpassmp}-\eqref{ass:n4} it follows that $\Phi( r^* + \eps r_0 ) = \Phi r^* + \eps \psi \ge J^* = T J^*$. From the definition of $\Gamma$ in \eqref{gamdef} and \cref{ubrem}, we know that $\Phi r^*+\eps \psi \geq \Gamma J^*\geq J^*$. The result follows by noting that $2\eps\psi\geq \Phi r^*+\eps\psi -J^*\geq \Gamma J^*-J^*\geq 0$.
\end{proof}
\begin{lemma}\label{tgmonotone}
For $J_1, J_2\in \R^n$ such that $J_1\leq J_2$, we have $\hg J_1\leq \hg J_2$.
\end{lemma}
\begin{proof}
Given $J\in \R^n$, let $\F_J\eqdef\{\,\Phi r\,: W^\top E \Phi r\geq W^\top HJ, r\in \N'\,\}\,$. Choose any $i\in \{1, \ldots, n\}$. Since $J_1\leq J_2$, from \cref{monotone} and \cref{grlpassmp}-\eqref{wassump} it follows that $W^\top H J_1\leq W^\top H J_2$. Hence, $\F_{J_2} \subset \F_{J_1}$ and thus $(\hg J_1)(i) \le (\hg J_2)(i)$.  Since $i$ was arbitrary, the result follows. 
\end{proof}
\begin{lemma}\label{maxnormmn}
Assume that $\hg: \R^n \to \R^n$ is monotone and 
that there exists some $\beta\in [0,1)$ such that for any $J\in \R^n$ and $t>0$,
\begin{align}
\label{eq:shiftmn}
\hg( J + t \psi) \le \hg J + \beta t \psi,
\end{align} 
for any $J\in \R^n$ and $t\ge 0$.
Then $\hg$ is a $\norm{\cdot}_{\mn}$ contraction with factor $\beta$.
\end{lemma}
\begin{proof}
First, we show that for any $t\ge 0$,  $J\in \R^n$,
$\hg( J - t \psi) \ge \hg J - \beta t \psi$ also holds.
To see this define $J' = J-t\psi$. Then, $J = J'+t\psi$, hence $\hg J \le \hg J' + \beta t \psi$. Reordering this inequality gives the result.
Let $\eps = \norm{J_1 - J_2}_{\mn}$, where $J_1,J_2\in \R^n$ are arbitrary.
Then $J_2 - \eps \psi \le J_1 \le J_2 + \eps \psi$. 
By the monotonicity of $\hg$,
$\hg(J_2 - \eps \psi) \le \hg J_1 \le \hg(J_2 + \eps \psi)$. 
Using~\eqref{eq:shiftmn}, we get 
$\hg J_2 - \beta \eps \psi \le \hg J_1 \le \hg J_2 + \beta \eps \psi$, i.e., $-\beta \eps \psi \le \hg J_1 - \hg J_2 \le \beta \eps \psi$, from which the result follows.
\end{proof}
\begin{corollary}\label{tmaxnormmn}
$T$ is a $\norm{\cdot}_{\mn}$-contraction with factor $\beta_{\psi}$.
\end{corollary}
%Returning to $\hg$, since we already now that $\hg$ is monotone (cf. \cref{gmonotone}), it remains to see that it satisfies \eqref{eq:shiftmn}:
\begin{lemma}\label{gshiftmn}
The operator $\hg$ satisfies \eqref{eq:shiftmn} with $\beta = \beta_\psi$.
%For any $J\in \R^n$, $t\in \R_+$, $\Gamma (J+ t \psi ) \le \Gamma J + \beta_{\psi} t \psi$.
\end{lemma}
\begin{proof}
By definition, for $1\le i \le n$, $(\hg (J+t\psi) )(i) = \min\{ e_i^\top \Phi r \,:\, W^\top E\Phi r \ge W^\top H(J+t\psi), r\in \N' \}$.
By \eqref{eq:psilin}, as $t>0$, $H(J+t\psi) \le HJ + t \beta_\psi \psi$ and hence $W^\top H(J+t\psi) \le W^\top (HJ + t \beta_\psi \psi)$. Thus,
$(\hg (J+t\psi) )(i) \le 
 \min\{ e_i^\top \Phi r \,:\, W^\top E\Phi r \ge W^\top HJ+t\beta_\psi \psi), r\in \N' \}$.
Now, using Lemma~\ref{lpsol} with $A=W^\top E \Phi$, $b=W^\top HJ$, $c=e_i$, $b_0=t\beta_\psi \psi$ 
and $x_0=t \beta_\psi r_0$, the statement follows since $A x_0 = b_0$ (from \cref{grlpassmp}-\eqref{lyap}) and $\N' = \N' + \alpha t r_0$ (from \cref{grlpassmp}-\eqref{ass:n4}).
%The result follows in a similar manner as the proofs for Lemmas~\ref{gshift} and ~\ref{tgshift} by using the result in Lemma~\ref{lpsol}.
\end{proof}
%From this result and \cref{maxnormmn}, we immediately get that $\Gamma$ is a contraction in $\norm{\cdot}_{\mn}$:
\begin{comment}
\begin{theorem}\label{gmaxcontramn}
The operator $\Gamma  \colon \R^n\ra \R^n$ is a contraction operator in $\norm{\cdot}_{\mn}$ with factor $\beta_{\psi}$.
\end{theorem}
\end{comment}
%In a similar way, one can also show that $\hg$ is also a contraction:
\begin{theorem}\label{hgmaxcontramn}
The operator $\hg:\R^n\to\R^n$  is a contraction operator in $\norm{\cdot}_{\mn}$ with factor $\beta_{\psi}$.
%$\hg$ is also a contraction map in the weighted $L_\infty$ norm.
\end{theorem}
\begin{proof}
Follows from \cref{maxnormmn,gshiftmn}.
\begin{comment}
We already know that $\hg$ is monotone. That $\hg$ satisfies~\cref{eq:shiftmn}
with $\beta = \beta_{\psi}$ follows similarly to the argument used in  \cref{tgshift}
with modifications similar to those introduced in the proof of \cref{gshiftmn}.
Then, \cref{gmaxcontramn} gives the desired result.
\end{comment}
\end{proof}
In what follows we denote by $\hv$ the unique fixed point of $\hg$, i.e., $\hv = \hg \hv\,$. 
We now show that vector $\hj$ dominates $\hv$:
\begin{lemma}\label{relation1}
The vectors $\hv,\hj$ obey $\hj\geq\hv$.
\end{lemma}
\begin{proof}
For $i\in \{1,\dots,n\}$, $c=e_i$ let $r_i$ be a solution to the GRLP in \eqref{grlp}
and define $V_0\in \R^n$ by $V_0(i)=\underset{j=1,\ldots,n}{\min}(\Phi r_j)(i)$, $1\le i \le n$.

It suffices to show that $V_1\eqdef \hg V_0 \le V_0 \le \hj$ since then the desired result follows
by defining $V_{n+1} = \hg V_n$, $n\ge 1$, noting that by \cref{tgmonotone}, $V_{n+1}\le V_{n}$ and by  \cref{tmaxnormmn}, $V_n \to \hv$.

Since $(\Phi r_j)(i) \ge (\Phi r_i)(i)$ also holds for any $1\leq i,j\leq n$ we have $V_0(i)  = (\Phi r_i)(i)$. Also, since $\hj(i) \ge (\Phi r_i)(i),1\leq i \leq n$ it follows that $\hj\geq V_0$. 
Now,  fix any $i$. 
We need to show that $V_1(i)=(\hg V_0)(i) = (\Phi r'_{e_i,V_0})(i) \leq V_0(i)$. 
By the definition of $r'_{e_i,V_0}$ we know that $(\Phi r'_{e_i,V_0})(i) \le (\Phi r)(i)$
holds for any $r\in \N$ such that $W^\top E \Phi r \ge W^\top H V_0$. 
Now it suffices to show that $r_i$ satisfies $W^\top E \Phi r_i \ge W^\top H V_0$. 
By definition, $r_i$ satisfies $W^\top E \Phi r_i \ge W^\top H \Phi r_i$.
Hence, by the monotone property of $H$ and \cref{grlpassmp}-\eqref{wassump} it is sufficient if $\Phi r_i \ge V_0$.
This however follows from the definition of $V_0$.
\end{proof}
\begin{lemma}\label{relation2}
The vectors $\hv,\tj$ obey $\tj\geq\hv$.
\end{lemma}
\begin{proof}
%The proof follows in a similar manner as the proof of \Cref{relation1}. To elaborate, 
Let $ r_1,  r_2,\ldots, r_n$ be solutions to the ALP in \eqref{alp} (with an additional constraint that the solution be restricted to $\N$) for $c=e_1, e_2,\ldots,e_n$, respectively,
and define $V_0\in \R^n$ by $V_0(i)=\underset{j=1,\ldots,n}{\min}(\Phi r_j)(i)$, $1\le i \le n$. The rest of the proof 
follows in the same manner as the proof of \cref{relation1}.
\end{proof}
\begin{lemma}\label{srw}
A vector
$\hat{r} \in \R^k$ is a solution to GRLP \eqref{grlp} iff it solves the following program:
\begin{align}\label{grlpeqprog}
\begin{split}
\min_{r\in \N}\, &||\Phi r-\hv||_{1,c}\,\mb
\text{s.t.}\mb \, W^\top E \Phi r\geq W^\top H \Phi r.
\end{split}
\end{align}
\end{lemma}
\begin{proof}
We know from Lemma~\ref{relation2} that $\hj = \Phi \hr \geq\hv$, and thus
the solutions to \eqref{grlp} do not change if we add the constraint $\Phi r \ge \hv$.
Now, under this constraint, minimizing $c^\top \Phi r$ is the same as minimizing 
\begin{align*}
||\Phi r-\hv||_{1,c}=\sum_{i=1}^n c(i) |(\Phi r)(i)-\hv(i)|=c^\top \Phi r-c^\top \hv\,.
\end{align*} 
\end{proof}
\begin{lemma}\label{srwalp}
A vector
$\tr \in \R^k$ is a solution to ALP \eqref{alp} iff it solves the following program:
\begin{align}\label{grlpeqprog}
\begin{split}
\min_{r\in \R^k}\, &||\Phi r-\hv||_{1,c}\,\mb
\text{s.t.}\mb \, \Phi r\geq  T \Phi r.
\end{split}
\end{align}
\end{lemma}
\begin{proof}
We know from Lemma~\ref{relation2} that $\tj = \Phi \tr \geq\hv$.
The rest of the argument follows in the same manner as the proof for \cref{srw}.
\end{proof}
\begin{lemma}\label{cmt1mn}
We have
\begin{align}
||J^*-\hv||_{\mn}
& \leq \frac{1}{{1-\beta_{\psi}}}\big(2||J^*-\Phi r^*||_{\mn}
\nn\\
& {}\qquad \qquad+||\Gamma J^*-\hg J^*||_{\mn}\big).
\end{align}
\end{lemma}
\begin{proof}
Recall that $\hg \hv=\hv$. By the triangle inequality,
\begin{align*}
||J^*-\hv||_{\mn}
& \leq ||J^*-\hg J^*||_{\mn}+||\hg J^*-\hg \hv||_{\mn}\\
&\leq ||J^*-\hg J^*||_{\mn}+\beta_\psi || J^*- \hv||_{\mn},
\end{align*}
and so by reordering and with another triangle inequality,
\begin{align}\label{jhv}
\norm{J^*-\hv}_{\mn} \nn
&\le \frac{\norm{ J^*-\hg J^*}_{\mn}}{1-\beta_\psi}\nn\\
&\le \frac{\norm{ J^*-\Gamma J^*}_{\mn}+\norm{\Gamma J^* - \hg J^*}_{\mn}}{1-\beta_\psi}\,.
\end{align}
The proof follows by applying \cref{bestbndmn} on \eqref{jhv}.
\end{proof}
We now recall Lemma~$5$ from Section 4.2 of \cite{ALP}. 
For this result, recall that $r_0 \in \R^k$ is the vector such that $\psi = \Phi r_0$.
\begin{lemma}\label{restate}
%Let $\psi$ be a Lyapunov function that belongs to the column span of $\Phi$ ,
For  $r \in \R^k$ arbitrary, let $r'$ be
\begin{align}
 r'= r+||J^*-\Phi r||_{\mn} \left(\frac{1+\beta_{\psi}}{1-\beta_{\psi}}\right)\, r_0.
\end{align}
Then $r'$ is feasible for the ALP in \eqref{alp}.
\end{lemma}
Recall that $\hv$ is the fixed point of $\hg$ and $\hj=\Phi \hr$ is the solution to the GRLP
\eqref{grlp}. 
\begin{theorem}\label{mt2mn}
We have
\begin{align}
||\hj-\hv||_{1,c}&\leq \frac{c^\top \psi}{1-\beta_\psi}(4||J^*-\Phi r^*||_{\mn}\nn
+||\Gamma J^*-\hg J^*||_{\mn}).
\end{align}
\end{theorem}
\begin{proof}
Let $\gamma=||J^*-\Phi r^*||_{\mn}$. 
Then, by choosing $r'$ as in Lemma~\ref{restate} we have
\begin{align}
||\Phi r'-J^*||_{\mn}\nn
&\leq ||\Phi r^*-J^*||_{\mn}+||\Phi r'-\Phi r^*||_{\mn}\nn\\
&=\gamma+\frac{1+\beta_\psi}{1-\beta_\psi}\gamma
	=\frac{2}{1-\beta_\psi}\gamma.
\label{part}
\end{align}
%We know that $\tr \in \N$ by \cref{grlpassmp}-\eqref{nassmp} and 
Now, $r'$ is feasible for the ALP in \eqref{grlp} by \cref{restate}. 
Then from \cref{srwalp} it follows that
%As a result, from \cref{srw} it follows that
\begin{align}
||\hj-\hv||_{1,c}
&\leq||\Phi \tr-\hv||_{1,c}\leq||\Phi r'-\hv||_{1,c}\nn\\
&=\sum_{s\in S}c(s)\psi(s)\frac{|\Phi r'(s)-\hv(s)|}{\psi(s)}\nn\\
&\leq c^\top \psi ||\Phi r'-\hv||_{\mn}\nn\\
&\leq c^\top \psi (||\Phi r'-J^*||_{\mn}+||J^*-\hv||_{\mn}).\nn
\end{align}
The result follows from \cref{cmt1mn} and \eqref{part}.
%\textbf{Main Result~$1$: Prediction Error bound in weighted $L_\infty$-norm}
\end{proof}
\begin{theorem}[Prediction error bound in $\norm{\cdot}_{\mn}$]
\label{cmt2mn}
%Let $\hj$, $\hv$, $r^*$ and $J^*$ be as in Theorem~\ref{mt2mn}, then
It holds that
\begin{align}\label{finalbndmn}
\begin{split}
||J^*-\hj||_{1,c}
&\leq\frac{c^\top\psi}{1-\beta_\psi}(6 ||J^*-\Phi r^*||_{\mn}\\
&+2||\Gamma J^*-\hg J^*||_{\mn}).
\end{split}
\end{align}
\end{theorem}
\begin{proof}
We have
\begin{align}
||J^*-\hj||_{1,c}\nn
&\leq||J^*-\hv||_{1,c}+||\hv-\hj||_{1,c}\nn\\
&\leq c^\top \psi ||J^*-\hv||_{\mn}+||\hv-\hj||_{1,c}.\nn
\end{align}
The result now follows from Lemma~\ref{cmt1mn} and Theorem~\ref{mt2mn}.
\end{proof}
We now bound the performance of the greedy policy $\hu$.
\begin{theorem}[Control Error Bound in $\norm{\cdot}_{\mn}$]
\label{polthe}
Let $\hu$ be the greedy policy with respect to the solution $\hj$ of the GRLP and $J_{\hu}$ be its value function.
% Let $r^*$ be as in Theorem~\ref{mt2mn}, then
Then,
\begin{align}\label{polthebnd}
%||J_{\hu}-\hj||_{1,c}
\norm{J^* - J_{\hu}}_{1,c}
&\leq 2\left(\frac{c^\top \psi}{(1-\beta_{\psi})^2}\right)\, \big( 2||J^*-\Phi r^*||_{\mn}
\nn\\&
+||\Gamma J^*-\hg J^*||_{\mn}+||\hj-\hg\hj||_{\mn}\big).
\end{align}
\end{theorem}
\begin{proof}
By the triangle inequality,
\begin{align*}
||J^*-J_{\hu}||_{1,c}&\leq ||J^*-\hj||_{1,c}+||J_{\hu}-\hj||_{1,c}\,.
\end{align*}
Let us now bound the second term on the right-hand side.
Since $\hu$ is greedy w.r.t. $\hj$, it holds that $T_{\hu} \hj = T \hj$.
Also, $T_{\hu} J_{\hu} = J_{\hu}$.
Hence, $J_{\hu} - \hj = T_{\hu} J_{\hu} - T_{\hu} \hj + T \hj - \hj
=\alpha P_{\hu} (J_{\hu}- \hj) + T\hj - \hj$.
Hence,
\begin{align}\label{polderv}
||J_{\hu}-\hj||_{1,c}\nn
&=||(I-\alpha P_{\hu})^{-1}(T\hj-\hj)||_{1,c}\nn\\
&\leq c^\top(I-\alpha P_{\hu})^{-1}|T\hj-\hj|\nn\\
&\leq c^\top (I-\alpha P_{\hu})^{-1} \,\psi\, ||T\hj-\hj||_{\mn}\nn\\
&\leq \frac{c^\top \psi}{1-\beta_{\psi}}||T\hj-\hj||_{\mn}\nn\\
%&\leq \frac{c^\top \psi}{1-\beta_{\psi}}||T\hj-TJ^* +J^*- \hj||_{\mn}\nn\\
&\leq \frac{c^\top \psi}{1-\beta_{\psi}}(||T\hj-TJ^*||_{\mn} +||J^*- \hj||_{\mn})\nn\\
&\leq \frac{c^\top \psi}{1-\beta_{\psi}}(1+\beta_{\psi})||J^*- \hj||_{\mn},
\end{align}
where in the second inequality, we use Jensen's inequality and $|T\hj - \hj|$ stands for the 
vector whose $i$th component is $|(T\hj)(i) - \hj(i)|$. Further, the last inequality follows
since $T$ is a $\norm{\cdot}_{\mn}$ contraction with factor $\beta_{\psi}$ as noted earlier.
%componentwise  $(I-\alpha P_{\hu})^{-1}$ is a positive operator for $x=(x_1,\ldots,x_n)^\top\in \R^n$, $|x|=(|x_1|,\ldots,|x_n|)^\top\in \R^n$. 
Hence,
\begin{align}
&||J^*-J_{\hu}||_{1,c}\nn\\
%&\leq ||J^*-\hj||_{1,c}+||J_{\hu}-\hj||_{1,c}\nn\\
&\leq c^\top \psi ||J^*-\hj||_{\mn}+c^\top \psi\frac{1+\beta_\psi}{1-\beta_\psi}||J^*- \hj||_{\mn}\nn\\
&=\frac{2c^\top \psi}{1-\beta_{\psi}}||J^*- \hj||_{\mn}.
\end{align}
Now in a manner similar to \cref{cmt2mn} we have
\begin{align}
||J^*- \hj||_{\mn}&\leq ||J^*- \hv||_{\mn}+||\hv -\hj||_{\mn}\nn
\end{align}
The result now follows by substituting the bound on $||J^*- \hv||_{\mn}$ from \cref{cmt1mn} and the fact that $||\hv-\hj||_{\mn}\leq \frac{1}{1-\beta_{\psi}}||\hj-\hg\hj||$.
\end{proof}
\begin{comment}
\begin{note}
By bounding $\etmn=||\Gamma J^*-J^*+J^*-\hg J^*||_{\mn}\leq 2||J^*-\Phi r^*||_{\mn}+||J^*-\hg J^*||_{\mn}$ 
(the inequality follows from \cref{bestbndmn}), 
we can loosen the bounds in \cref{cmt2mn} and \cref{polthe} to
\begin{align}
\label{loose1}
||J^*-\hj||_{1,c}&\leq\frac{c^\top\psi}{1-\beta_\psi}(10 ||J^*-\Phi r^*||_{\mn}
\nn\\&
+2||J^*-\hg J^*||_{\mn}).\\
\label{loose2}
||J^* - J_{\hu}||_{1,c}&\leq 2\left(\frac{c^\top \psi}{1-\beta_{\psi}}\right)^2 \,\big(10 ||J^*-\Phi r^*||_{\mn}
\nn\\&
+2||J^*-\hg J^*||_{\mn}\big).
\end{align}
Here the term $||J^*-\hg J^*||$ in \eqref{loose1} and \eqref{loose2} captures the error due to the use of both $\Phi$ and $W$. Though, \eqref{loose1} and \eqref{loose2} might be loser bounds than \eqref{finalbndmn} and \eqref{polthebnd} respectively, the advantage of this bound is that it captures the error due to function approximation as well as constraint reduction in a direct manner.
\end{note}
\end{comment}


