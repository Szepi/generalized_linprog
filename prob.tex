\section{Problem Setup}
We now present the linear programming (LP), approximate linear programming (ALP) and the generalized reduced approximate linear programming (GRLP) respectively. In what follows, $c>0\in \R^n$ is a probability distribution.
It is well known, the optimal value function $J^*$ can be obtained by solving the following linear program:
\begin{align}\label{mdplp}
\begin{split}
\min_{J\in \R^n}\, &c^\top J\,\,\text{s.t.}\\
%\mb
J(s)&\geq g_a(s)+\alpha\sum_{s'}p_a(s,s')J(s'),\,\,
\forall (s,a)\in S\times A,
\end{split}
\end{align}
\begin{comment}
When the MDP has a large number of states, it is difficult to solve for $J^*$ using either the linear program \eqref{mdplp} or other full state representation methods such as value iteration or policy ite
ration \cite{BertB}. A practical solution is to resort to function approximation. Linear function approximation, wherein the solution is searched in the subspace spanned by the column vectors of a given feature matrix $\Phi$.\par
\end{comment}
The approximate linear program (ALP) is obtained by making use of LFA in the LP, i.e., by introducing the new variables $r\in \R^k$ and adding the extra constraint $J=\Phi r$ in \eqref{mdplp} with $\Phi \in \R^{n\times k}$ \citep{SchSei85}.
By substitution, this leads to
\begin{align}\label{alp}
\begin{split}
\min_{r\in \R^k}\, &c^\top \Phi r\,\,\,
\text{s.t.}\mb \Phi r\geq T \Phi r,
\end{split}
\end{align}
where $J\geq TJ$ is a shorthand for the $nd$ constraints in \eqref{mdplp} and $\Phi$ is a feature matrix whose first column is $\one$.
The Generalized Reduced Linear Program is given as:
\begin{align}\label{grlp}
\begin{split}
&\underset{r\in \N\subset R^k}{\min}\, \, c^\top \Phi r\,\,\,\,\\
&\text{s.t.}\mb  \,W^\top E\Phi r\geq W^\top H \Phi r,
\end{split}
\end{align}
Unless specified otherwise we use $\hr$/$\tr$ to denote an arbitrary solution to the GRLP/ALP. In similar spirit, and we let $\hj=\Phi \hr$/$\tj=\Phi \tr$ to denote the corresponding value functions and $\hu$/$\tu$ to denote the greedy policy w.r.t. $\hj$/$\tj$.\par
\textbf{Remarks:}
\begin{enumerate}
\item When the MDP has a large number of states, it is difficult to solve for $J^*$ using either the linear program \eqref{mdplp} or other full state representation methods such as value iteration or policy iteration \cite{BertB}. A practical solution is to resort to function approximation. Linear function approximation, wherein the solution is searched in the subspace spanned by the column vectors of a given feature matrix $\Phi$. For $k=n$ and $\Phi=I_{n\times n}$, the ALP is same as the LP.
\item $W \in \R^{nd\times m}_+$ is a matrix with all positive entries which specifies the linear combination of constraints. $\N$ is a compact set such that $\tr \in \N$. This additional constraint set ensures the boundedness of the solution. For $m=nd$ and $W=I_{nd\times nd}$, the GRLP is same as the ALP.
\end{enumerate}
\FloatBarrier
%!TEX root =  autocontgrlp.tex
\begin{figure}[h!]
\centering
%\resizebox{columnwidth}{2cm}{
%\begin{tikzpicture}[domain=-10:7.7,scale=0.6,font=\small,axis/.style={very thick, ->, >=stealth'}]
\begin{tikzpicture}[domain=-10:7.7,scale=0.5,font=\normalsize,axis/.style={very thick, ->, >=stealth'}]
%\draw[line,thick,->] (-1,-0.625)--(4,-0.625);
%\draw[line,thick,->] (0,-1)--(0,4);
%\draw[line,thick,-](-0.2,-0.6)--(1,3);
\draw[line,thick,-](0.5,3.5)--(1,1);
\draw[line,thick,-](1,1)--(3,1.5);
\draw[line,thick,-](3,1.5)--(4,2.1);
\node[](one) at (2,2.3){\text{$J\geq TJ$}};
\node[rotate=-45](seven) at (-0.5,2.3){\text{\tiny $W^\top J\geq W^\top TJ$}};
\node[](two) at (-0.3,-0.5){\text{$\Phi r$}};
\node[](three) at (0.7,0.7){\text{$J^*$}};
%\draw[line,thick,-](0,0)--(4,2.5);
 \draw [ultra thick, draw=white, fill=gray, opacity=0.5]
       (0.5,3.5)--(1,1)--(3,1.5)--(4,2.1) -- cycle;
\draw[line,thick,-](-1,-0.625)--(3,1.8750);
 \fill (1,1)  circle[radius=2pt];
 \fill (2,1.25)  circle[radius=2pt];
 \fill (0.5,0.3125)  circle[radius=2pt];
\draw[line,dashed,-](-1,0.1)--(6,1);
\draw[line,dashed,-](-1,0.1)--(-2,5);
 \draw [ultra thick, draw=white, fill=gray, opacity=0.2]
       (-1,0.1)--(6,1)--(-2,5) -- cycle;
\node[] (four) at (2,0.8){\text{$\tilde{J}_c$}};
\node[] (six) at (0.5,-0.1){\text{$\hat{J}_c$}};
\end{tikzpicture}
%}
\caption{
%\normalsize 
The outer lightly shaded region corresponds to GRLP constraints and the inner dark shaded region corresponds to the original constraints. The main contribution of the paper is to bound $||J^*-\hat{J}_c||$.}
\label{cartoon}
\end{figure}


\Cref{cartoon} which shows the solutions to the LP, ALP and GRLP respectively. The erorr in the ALP solution has already been studied in \cite{ALP}. Our objective is to study the extra source of error due to constriant approximation.
\begin{comment}
In addition, we answer the following questions related to constraint reduction in ALP that have so far remained open. \\
$\bullet$ As a natural generalization of the RLP, what happens if we define a generalized reduced linear program (GRLP) whose constraints are positive linear combinations of the original constraints of the ALP?\\
$\bullet$ Unlike \cite{CS} which provides error bounds for a specific RLP formulated using an idealized sampling distribution, is it possible to provide error bounds for any GRLP (and hence any RLP)?
In this paper, we address both of the questions above.
\end{comment}
