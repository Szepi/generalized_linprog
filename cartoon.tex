%!TEX root =  autocontgrlp.tex
\begin{figure}[h!]
\centering
%\resizebox{columnwidth}{2cm}{
%\begin{tikzpicture}[domain=-10:7.7,scale=0.6,font=\small,axis/.style={very thick, ->, >=stealth'}]
\begin{tikzpicture}[domain=-10:7.7,scale=0.5,font=\normalsize,axis/.style={very thick, ->, >=stealth'}]
%\draw[line,thick,->] (-1,-0.625)--(4,-0.625);
%\draw[line,thick,->] (0,-1)--(0,4);
%\draw[line,thick,-](-0.2,-0.6)--(1,3);
\draw[line,thick,-](0.5,3.5)--(1,1);
\draw[line,thick,-](1,1)--(3,1.5);
\draw[line,thick,-](3,1.5)--(4,2.1);
\node[](one) at (2,2.3){\text{$J\geq TJ$}};
\node[rotate=-45](seven) at (-0.5,2.3){\text{\tiny $W^\top J\geq W^\top TJ$}};
\node[](two) at (-0.3,-0.5){\text{$\Phi r$}};
\node[](three) at (0.7,0.7){\text{$J^*$}};
%\draw[line,thick,-](0,0)--(4,2.5);
 \draw [ultra thick, draw=white, fill=gray, opacity=0.5]
       (0.5,3.5)--(1,1)--(3,1.5)--(4,2.1) -- cycle;
\draw[line,thick,-](-1,-0.625)--(3,1.8750);
 \fill (1,1)  circle[radius=2pt];
 \fill (2,1.25)  circle[radius=2pt];
 \fill (0.5,0.3125)  circle[radius=2pt];
\draw[line,dashed,-](-1,0.1)--(6,1);
\draw[line,dashed,-](-1,0.1)--(-2,5);
 \draw [ultra thick, draw=white, fill=gray, opacity=0.2]
       (-1,0.1)--(6,1)--(-2,5) -- cycle;
\node[] (four) at (2,0.8){\text{$\tilde{J}_c$}};
\node[] (six) at (0.5,-0.1){\text{$\hat{J}_c$}};
\end{tikzpicture}
%}
\caption{
%\normalsize 
The outer lightly shaded region corresponds to GRLP constraints and the inner dark shaded region corresponds to the original constraints. The main contribution of the paper is to bound $||J^*-\hat{J}_c||$.}
\label{cartoon}
\end{figure}

