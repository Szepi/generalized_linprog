%!TEX root =  autocontgrlp.tex
\section{Constraint sampling}
One way to reduce the number of constraints is to randomly choose a subset of them \cite{CS},
leading to the so-called reduced linear program (RLP). \todoc{How about the results of Calafiore and Campi (2005, Math. Prog.) and others reviewed in \cite{CS}? Any follow-up we missed?}
While the objective function of the RLP is the same as that of the ALP, 
the RLP has only $m \ll nd$ constraints sampled from the original ALP according to a given probability distribution:
\begin{align}\label{rlp}
\min_{r\in \N}\, &c^\top \Phi r\nn\\
\text{s.t.}\mb &(\Phi r)(s)\geq g_a(s)+\alpha\sum_{s'}p_a(s,s')(\Phi r)(s'), \nn\\
&\forall (s,a) \in \I.
\end{align}
where $\I$ is the index set containing $m$ sampled state-action pairs and the extra constraint set $\N$ is introduced to guarantee a bounded solution. Note that apart from $r\in \N$ the feasible set of the RLP is a superset of the feasible set of the ALP. 
\begin{comment}
The index set $\I$ in \eqref{rlp} naturally gives rise to the definition of an $nd\times m$ constraint sampling matrix $\M$:
\begin{definition}\label{csampmat}
Let $\I=\{(s_1,a_1),\ldots,(s_m,a_m)\}$ be the sampled state-action pairs and let $q_i \eqdef s_i+(a_i-1)\times n$, $i=1,\ldots,m$. Define the constraint sampling matrix $\M \in \R^{nd\times m}$ associated with the index set $\I$ by
\begin{align}
\M(i,j)
=
\begin{cases}
1, & \text{ if } q_i = j\,;\\
0, & \text{ otherwise}.
\end{cases}
\end{align}
\end{definition}
The RLP can then be represented in short as
\begin{align}\label{rlpshort}
\begin{split}
\min_{r\in \N}\, &c^\top \Phi r\\
\text{s.t.}\mb & \M^\top E \Phi r \geq \M^\top H \Phi r\,.
\end{split}
\end{align}
The error bounds for the RLP can be found in \cite{CS} and the RLP has been shown to perform well in experiments \cite{ALP,CS,CST} in various domains such as Tetris and in network of queues. An alternative to constraint sampling is to introduce function approximation in the dual variables of the ALP \cite{ALP-Bor,dolgov}. While such an approach has been applied in practice, there are no known theoretical guarantess for function approximation of the dual variables.\par
In the next section, we present the generalized reduced linear program (GRLP) which is the main object of study in this paper.
\end{comment}

