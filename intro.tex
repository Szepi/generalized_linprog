%!TEX root =  autocontgrlp.tex
\section{Introduction}
Markov decision processes (MDPs) is a powerful mathematical framework to study optimal sequential decision making problems that arise in science and engineering. The so-called dynamic programming methods find an optimal policy by computing the optimal \emph{value-function} ($J^*$), a vector whose dimension is the number of states. MDPs with small number of states can be solved easily by conventional solution methods such as value-, or policy-iteration, or linear programming (LP) \cite{BertB}. 

A practical way to tackle the issue of large number of states is to compute an approximate value function $\tilde{J}$ instead of $J^*$. Here, success depends on the quality of approximation, i.e., on the quantity $||J^*-\tilde{J}||$ for an appropriately chosen norm. The most widely used method for approximation is the linear function approximation (LFA), i.e., let $\tilde{J}=\Phi r^*$, where $\Phi$ is a feature matrix and $r^*$ is a weight vector to be computed. Dimensionality reduction is achieved by choosing $\Phi$ to have fewer columns in comparison to the number of states, making it possible to work with large MDPs where even storing one value per state would be impossible in the computer's main memory.

The \emph{approximate linear program} (ALP) \cite{ALP,CS,SALP,ALP-Bor} and its variants introduce linear function approximation in the linear programming formulation. A critical shortcoming of the ALP is that the number of constraints are of the order of the size of the state space, making the vanilla version of the ALP intractable. A way out is to employ choose a subset of constraints random and drop the rest, thereby formulating a \emph{reduced linear program} (RLP). The performance analysis of the RLP can be found in \cite{CS} and the RLP has also been shown to perform well in experiments \cite{ALP,CS,CST}. An alternative approach to handle the issue of large number of constraints is to employ function approximation in the dual variables of the ALP \cite{ALP-Bor,dolgov}, an approach that was empirically found to be useful. However, to this date, there exist no theoretical guarantees bounding the loss in performance resulting from such an approximation.

In this paper, we generalize the RLP to define a generalized reduced linear program (GRLP) which has a tractable number of constraints obtained as positive linear combinations of the original constraints of the ALP. 
The salient aspects of our contribution are listed below:
\begin{enumerate}
		\item We develop novel analytical machinery to relate $\hat{J}$, the solution to the GRLP, and the optimal value function $J^*$ by bounding the prediction error $||J^*-\hj||$ (\Cref{cmt2mn}). 
		\item We also bound the performance loss due to using the one-step greedy policy $\hu$ (greedy with respect to $\hj$) as a function of $||J^*-J_{\hu}||$ (Theorem~\ref{polthe}).
		\item Our analysis is based on two novel $\max$-norm contraction operators and our results hold \emph{deterministically}, as opposed to the results on RLP \cite{SALP,CS}, where the guarantees have a probabilistic nature.
		\item Our results on the GRLP are the first to theoretically justify linear function approximation of Langrangian (dual) variables underlying the constraints.
		\item A numerical example in controlled queues is provided to illustrate the theory.
\end{enumerate}

