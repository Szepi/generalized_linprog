%!TEX root =  autocontgrlp.tex
\section{Approximate Least Upper Bound Projection}\label{sec:alubp}
We define an approximate least upper bound (ALUB) projection operator which has a structure similar to the GRLP and is an approximation to the LUB operator.
Recall that $W\in \R^{nd\times m}$ is the matrix in \eqref{grlp} which is assumed to satisfy Assumption~\ref{wassump}.
\begin{definition}\label{alubpop}
Given $J\in \R^n$ and the nonnegative valued vector $c\in \R^n_+$, define $r_{c,J}'$ to be the solution to 
\begin{align}\label{alubplp}
 \underset{r\in \R^k}{\min}& \,\mb c^\top \Phi r\,,\nn\\
 \text{s.t.}& \,\,\, W^\top E \Phi r\geq W^\top HJ\,,\qquad r \in \N\,.
\end{align}
Then, for $J\in \R^n$, $\hg J$, the approximate least upper bound (ALUB) projection of $J$ is defined as 
\begin{align}\label{tgamdef}
(\hg J)(i)\eqdef(\Phi \har_{e_i,J})(i), \mb i=1,\ldots,n\,.
\end{align}
\end{definition}
\noindent 
Note that under our assumptions on $\N$, $\hg$ is well-defined.
We now show that $\hg$ satisfies the monotone property and is linear along $[\one]$-rays with a factor smaller than one and hence it will follow that it is a contraction.
\begin{lemma}\label{tgmonotone}
For $J_1, J_2\in \R^n$ such that $J_1\leq J_2$, we have $\hg J_1\leq \hg J_2$.
\end{lemma}
\begin{proof}
By \cref{wassump}, $W^\top H: \R^n \to \R^m$ is monotone.
The proof then follows along the lines of Lemma~\ref{gmonotone}.
\end{proof}
\begin{lemma}\label{tgshift}
Let $J\in \R^n$ and $t\in \R$ be a constant. Then $\hg (J+t\one)=\hg J+\alpha t\one$.
\end{lemma}
\begin{proof}
The statement follows from Assumptions \ref{one}, \ref{wassump} and \ref{ass:n}, 
as well as Lemma~\ref{lpsol} using arguments along the lines of Lemma~\ref{gshift}. 
In particular, consider the $i^{th}$ linear program corresponding to $\hg J$ and $\hg (J+t\one)$. 
Now, the result follows by letting $A=W^\top E \Phi$, $b=W^\top H J$, 
$c=e_i$, $b_0=\alpha t \mathbf{1}$, 
$x_0=\alpha t e_1$,
noting that thanks to
\cref{one} and \ref{wassump}, $A x_0 = b_0$ and that thanks to
\cref{ass:n} \eqref{ass:n1}, $\N = \N + \alpha t e_1$.
\end{proof}
From these results, by~\cref{maxnorm} it immediately follows that $\hg$ is an $\alpha$-contraction with respect to the $\max$-norm:
\begin{theorem}\label{tgmaxcontra}
The operator $\hg \colon \R^n\ra \R^n$ obeys the $\max$-norm contraction property with factor $\alpha$.
\if0
 and the following iterative scheme based on the ALUB projection operator $\hg$, see \eqref{apvi}, converges to a unique fixed point $\hv$.
\begin{align}\label{apvi}
V_{n+1}&=\hg V_n,\mb n\geq 0.
\end{align}
\fi
\end{theorem}
\begin{proof}
As noted, this is immediate from \cref{tgmonotone,tgshift} and \cref{maxnorm}.
%Follows along similar lines as the proof of Theorem~\ref{gmaxcontra}.
\end{proof}
In what follows we denote by $\hv$ the unique fixed point of $\hg$:
\begin{align*}
\hv = \hg \hv\,.
\end{align*}
Recall that $\hj$ denotes the solution $\hj$ of the GRLP \eqref{grlp}. 
The vector $\hj$ dominates the vector $\hv$:
\begin{lemma}\label{relation2}
The vectors $\hv,\hj$ obey $\hj\geq\hv$.
\end{lemma}
\begin{proof}
The proof follows in a similar manner as the proof of Lemma~\ref{relation1}. To elaborate, let $\Phi r_1, \Phi r_2,\ldots,\Phi r_n$ be solutions to the GRLP in \eqref{grlp} for $c=e_1, e_2,\ldots,e_n$ respectively. Fix some other $c\in [0,1]^n$, $\sum_i c(i)=1$. Now consider the iterative scheme in \eqref{apvi} with $V_0(i)=\underset{j=1,\ldots,n}{\min}(\Phi r_j)(i)$.\par
We need to show $V_1\le V_0 \le \hj$ and then the desired result follows from Lemma~\ref{tgmonotone}. Since $(\Phi r_j)(i) \ge (\Phi r_i)(i)$ also holds for any $1\leq i,j\leq n$ we have $V_0(i)  = (\Phi r_i)(i)$. Also, since $\hj(i) \ge (\Phi r_i)(i),1\leq i \leq n$ it follows that $\hj\geq V_0$. Now, for showing that $V_1 \le V_0$ holds, fix any $i$. We need to show that $V_1(i)=(\hg V_0)(i) = (\Phi r_{e_i,V_0})(i) \leq V_0(i)$. By the definition of $r_{e_i,V_0}$ we know that $(\Phi r_{e_i,V_0})(i) \le (\Phi r)(i)$
holds for any $r$ such that $W^\top E \Phi r \ge W^\top H V_0$. Thus, it suffices to show that $r_i$ satisfies this latter inequality, i.e., $W^\top E \Phi r_i \ge W^\top H V_0$. For this, it clearly sufficient if $\Phi r_i \ge V_0$. This however directly follows from the definition of $V_0$.
\begin{comment}
It is clear from the definition of $V_0$ that $\hj(i),(\Phi r_j)(i)\geq(\Phi r_i)(i)$, $i=1,\ldots,n$.
In particular, it follows that $V_0(i) =( \Phi r_i)(i)$ and so
$\Phi r_i \geq V_0$.
From the monotone property of $T$ (viz. $H$) we have 
$H\Phi r_i\geq H V_0$ and also, by minimizing component-wise
$E\Phi r_i\geq H\Phi r_i\geq HV_0$. Thus,
$E V_0\geq H V_0$. 
\todoc[inline]{I lost why this was needed.
I think the proof would be much easier to understand if we introduced $r_{c,J} = \argmin_{r} \{ c^\top \Phi r \,:\, W^\top E \Phi r \ge W^\top H J\}$. We could start by saying we need $V_1\le V_0 \le \hj$ because then Lemma~\ref{tgmonotone} will give the desired result. Next, $V_0\le \hj$ holds because on the one hand, $\hj(i) \ge (\Phi r_i)(i)$ for any $i$, while on the other hand,
$(\Phi r_j)(i) \ge (\Phi r_i)(i)$ also holds for any $i,j$, hence $V_0(i)  = (\Phi r_i)(i)$. 
Now, for showing that $V_1 \le V_0$ holds, fix any $i$. We need to show that $(\hg V_0)(i) = \min_j (\Phi r_{e_j,V_0})(i) = (\Phi r_{e_i,V_0})(i)$ is less than $V_0(i)$. By the definition of $r_{e_j,V_0}$ we know that $(\Phi r_{e_i,V_0})(i) \le (\Phi r)(i)$
holds for any $r$ such that $W^\top E \Phi r \ge W^\top H V_0$. Thus, it suffices to show that $r_i$ satisfies this latter inequality,
i.e., $W^\top E \Phi r_i \ge W^\top H V_0$. For this, it clearly sufficient if $\Phi r_i \ge V_0$. This however directly follows 
from the definition of $V_0$.
}
Since $V_1=\hg V_0$, from the definition of $\hg$ in \eqref{gamdef} and the construction of $V_0$, we have $V_0\geq V_1$, and recursively $V_{n}\geq V_{n+1}, \mb n\geq 0$. So it follows that $\hj\geq V_0\geq V_1\ldots\geq \hv$.
\end{comment}
\end{proof}
The next result bounds the distance between $J^*$ and $\hv$ in terms of the distance of $\bj = \Gamma J^*$ and $J^*$, and that between $\bj$ and $\hg J^*$.
\begin{theorem}\label{mt1}
We have
\begin{align}
||J^*-\hv||_\infty\leq \frac{||J^*-\Gamma J^*||_\infty+||\Gamma J^*-\hg J^*||_\infty}{1-\alpha}.
\end{align}
\end{theorem}
\begin{proof}
By the triangle inequality,
\begin{align}
||J^*-\hv||\leq ||J^*-\hg J^*||+||\hg J^*-\hg \hv||\leq ||J^*-\hg J^*||+\alpha || J^*- \hv||
%||J^*-\hv||_\infty&\leq ||J^*-V_0+V_0-V_1+V_1\ldots-\hv||_\infty\nn\\
%&\leq ||J^*-V_0||_\infty+||V_0-V_1||_\infty+||V_1-V_2||_\infty+\ldots\nn\\
%&=||J^*-V_0||_\infty+||\hg J^*-\hg V_0||_\infty+\ldots\nn\\
%&\leq (\epsilon+\beta)+\alpha(\epsilon+\beta)+\ldots\nn\\
%&=\frac{\epsilon+\beta}{1-\alpha}
\end{align}
and so by reordering and another triangle inequality,
\begin{align*}
\norm{J^*-\hv}_{\infty} \le \frac{\norm{ J^*-\hg J^*}_{\infty}}{1-\alpha}
\le \frac{\norm{ J^*-\Gamma J^*}_{\infty}+\norm{\Gamma J^* - \hg J^*}_{\infty}}{1-\alpha}\,.
\end{align*}
\end{proof}
\noindent Combining this result with \cref{bestbnd} gives the following corollary:
\begin{corollary}\label{cmt1}
%Let $\hv$, $\bj$ be as in Theorem~\ref{mt1} and let $r^*\eqdef\argmin_{r\in \R^k}||J^*-\Phi r||_\infty$, then
We have
\begin{align}
||J^*-\hv||_\infty\leq \frac{2||J^*-\Phi r^*||_\infty+||\Gamma J^*-\hg J^*||_\infty}{1-\alpha}.
\end{align}
\end{corollary}
%\begin{proof}
%The result is obtained by using Lemma~\ref{bestbnd} to replace the term $||J^*-\bj||_\infty$ in Theorem~\ref{mt1}.
%\end{proof}
