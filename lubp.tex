%!TEX root =  autocontgrlp.tex
\section{Least Upper Bound Projection}\label{sec:lubp}
Let $e_j$ denote the $j$th basis vector in the standard Euclidean bases of $\R^n$.
We make the following assumption on $\N$:
\begin{assumption}
\label{ass:n}
The set $\N$ satisfies the following conditions:
\begin{enumerate}[(i)]
\item \label{ass:n1}
For any $t\in \R$, $t e_1 + \N \eqdef \{ r + t e_1 \,:\, r\in \N \} = \N$, i.e., 
$\N$ is invariant to shifts by vectors of the form $t e_1$.
\item \label{ass:n2}
For $2\le k \le n$, $\sup_{r\in \N} | e_k^\top r | <+\infty$.
\item \label{ass:n3}
For each $1\le i \le n$, there exists a solution $r_i\in \R^k$ to the ALP \eqref{alp} with $c = e_i$ such that
$r_i\in \N$.
\end{enumerate}
\end{assumption}
The least upper bound (LUB) projection operator $\Gamma \colon \R^n \ra\R^n$ is defined as below:
\begin{definition}\label{lubpop}
Given $J\in \R^n$ and the nonnegative valued vector $c\in \R^n_+$, define $r_{c,J}$ to be the solution to 
\begin{align}
\label{lubplp}
\begin{split}
 \underset{r\in \R^k}{\min} &\,\, c^\top \Phi r\,,\\
 \text{s.t.}& \mb \Phi r\geq  TJ\,, \qquad r\in \N\,.
 \end{split}
\end{align}
Then, for $J\in \R^n$, $\Gamma J$,
the least upper bound projection of $J$ is defined as 
\begin{align}\label{gamdef}
(\Gamma J)(i)\eqdef(\Phi r_{e_i,J})(i),\quad i=1,\ldots,n\,.
\end{align}
\end{definition}
%Note that $r_{e_j}$ depends on $J$, but this dependence is suppressed to reduce clutter. 
\begin{remark}\label{lubremark}
The following hold: 
\begin{enumerate}
\item Thanks to \cref{ass:n} \eqref{ass:n1} and \eqref{ass:n2}, 
for arbitrary $J$ and arbitrary \emph{nonnegative valued} $c$,
 $\Phi r_{c,J}$ is uniquely defined.
Indeed, although $r_{c,J}$ might not be unique, but $\Phi r$ where $r$ belongs to the feasible
 region underlying \eqref{lubplp}, is bounded from below, hence  $\Phi r_{c,J}$ is uniquely defined.
It also follows that $\Gamma$ is well-defined.
\item Assuming that $\N$ is polyhedral, 
	the definition of LUB operator $\Gamma \colon \R^n \ra \R^n$ involves $n$ associated linear programs.
\item It holds that $\Gamma J\geq TJ$ (this follows from the fact that if $a\geq c$ and $b\geq c$, then $\min(a,b)\geq c$, where $a, b, c \in \R$).
\item Given $J\in \R^n$, define 
\begin{align*}
\F_J\eqdef\{\,\Phi r\,|\,\Phi r\geq TJ, r\in \N\,\}\,.
\end{align*}
Thus $\F_J$ contains vectors in the span of $\Phi$ that upper bound $TJ$. 
Further, since $(\Gamma J)(i) = \min\{ V(i) \,|\, V\in \F_J \}$, it also follows that $\Gamma J \le V$ holds for any $V\in \F_J$.
%By fixing $c$ in the linear program in \eqref{lubplp} we select a unique vector $\Phi r_c \in \F$. The LUB projection operator $\Gamma$ picks $n$ vectors $\Phi r_{e_i},i=1,\ldots,n$ from the set $\F$ and $\Gamma J$ is obtained by computing their component-wise minimum.
\item Even though $\Gamma J$ does not belong to the span of $\Phi$, $\Gamma J$ collates the various best upper bounds that can be obtained via the linear program in \eqref{lubplp}.
\begin{comment}
\item The LUB operator $\Gamma$ in \eqref{gamdef} bears close similarity to the ALP in \eqref{alp}.
In fact, it is not hard to observe that $(\Gamma J)(i) = (\Phi r_{e_i})(i)$ for any $1\le i \le n$; the reason $\Gamma$ is defined the way it is so that this resembles to \eqref{alp} will be clear. 
\end{comment}
\end{enumerate}
\end{remark}
%\begin{definition}\label{bestproj}
%\end{definition}
We now characterize the LUB projection operator $\Gamma$ in the following lemmas. 
%As mentioned earlier, the error analysis depends on two $\max$-norm contraction operators the first of which is $\Gamma$. The important result of this section is Theorem~\ref{fxpres} and it relates the fixed point $\tv$ of $\Gamma$ to $J^*$.
In what follows, we let $r^*\in \R^k$, $\bj\in \R^n$ be defined by
\begin{align*}
r^* & \eqdef \argmin_{r\in R^k}||J^*-\Phi r||_\infty\,\\
\bj & \eqdef \Gamma J^*\,.
\end{align*}
If $r^*$ is non-unique, one may choose any minimizer.
We will now show that $\Gamma$ is a contraction and that its fixed point is close to $J^*$ as long as $\bj$ is closed to $J^*$.
The following lemma shows that this latter always holds as long as $\Phi$ is well-chosen:
\begin{lemma}\label{bestbnd}
It holds that
\begin{align}
||J^*-\bj||_\infty\leq 2||J^*-\Phi r^*||_\infty\,.
\end{align}
\end{lemma}
\begin{proof}
By our earlier remark, 
\begin{align} \label{eq:jtjgj}
J^* = T J^* \le \Gamma J^*\,.
\end{align}
Define $\eps = \norm{J^* - \Phi r^*}_{\infty}$. Then, $\Phi(r^* + \eps e_1) = \Phi r^* + \eps \one \ge J^* = TJ^*$.
Hence, $\Phi(r^* + \eps e_1) \in \F_{J^*}$ and thus
by Remark~\ref{lubremark}, $\Phi(r^* + \eps e_1) \ge \Gamma J^*$. This, together with~\eqref{eq:jtjgj} means that
$J^* \le \Gamma J^* \le \Phi r^* + \eps \one$ and thus
$0 \le \Gamma J^* - J^* \le \Phi r^* - J^* + \eps \one \le 2 \eps \one$. Taking maximum over the components gives the result.
\if0
The result follows from the definition of $\Gamma$ in \eqref{gamdef} and the construction of $V_0$, Assumption~\ref{one}, and the fact that $\Phi r^*+||J^*-\Phi r^*||_\infty \one\geq TJ^*$. To see this, note that
\begin{align}
\Gamma J^*=\hj \geq J^*,\nn \text{ and }
\Phi r^* +||J^*-\Phi r^*||_\infty\geq TJ^*= J^*.\nn
\end{align}
Thus,
\begin{align}
\Phi r^* +||J^*-\Phi r^*||_\infty\geq \Gamma J^*\geq TJ^*.
\end{align}
\fi
\end{proof}

We now show that $\Gamma$ is a contraction. 
By \cref{maxnorm}, it suffices to show that $\Gamma$ is monotone and linear along $[\one]$-rays.
We start with monotonicity.
\begin{lemma}\label{gmonotone}
For $J_1, J_2\in \R^n$ such that $J_1\geq J_2$, we have $\Gamma J_1\geq \Gamma J_2$.
\end{lemma}
\begin{proof}
Choose any $i\in \{1,\ldots,n\}$. 
Since $J_1\geq J_2$, we have $TJ_1\geq TJ_2$.
Hence, $\F_{J_1} \subset \F_{J_2}$ and thus $(\Gamma J_1)(i) \ge (\Gamma J_2)(i)$. 
Since $i$ was arbitrary, the result follows.
\end{proof}
\begin{lemma}\label{lpsol}
Let $A\in \R^{u\times v}$, $b,c\in \R^u$ and $b_0=Ax_0$ for 
some $x_0 \in \R^v$, $\N \subset \R^v$ such that $\N =x_0+ \N$. Then
\begin{align}
\min\{c^\top Ax:Ax\geq b+b_0, x\in \N\} =\min\{c^\top A y:Ay \geq b, y \in \N \}+c^\top b_0.
\end{align}
\end{lemma}
\begin{proof}
The claim follows by the change of variables $y := x-x_0$.
\end{proof}
\begin{lemma}\label{gshift}
Let $J\in \R^n$ and $t\in \R$ be a constant.
Then, $\Gamma (J+t\one) = \Gamma J + \alpha t \one$.
\end{lemma}
\begin{proof}
Consider the $i^{th}$ linear programs associated with $\Gamma J$ and $\Gamma (J+ t\one)$. 
The result follows by using Lemma~\ref{lpsol} with $A=\Phi$, $b=TJ$, $c=e_i$, $b_0=\alpha t\mathbf{1}$ 
and $x_0=\alpha t e_1$, noting that by our assumption on $\N$, $\N = \N + \alpha t e_1$.
\end{proof}
\begin{theorem}\label{gmaxcontra}
The operator $\Gamma  \colon \R^n\ra \R^n$ obeys the $\max$-norm contraction property with factor $\alpha$.
\end{theorem}
\begin{proof}
The statement follows immediately from Lemmas~\ref{gmonotone} and~\ref{gshift}, and \cref{maxnorm}.
\end{proof}
\if0
\begin{corollary}
Let $V_0$ be arbitrary and $(V_n)$ be defined recursively by
\begin{align}\label{pvi}
V_{n+1}&=\Gamma V_n, \quad n\geq 0\,.
\end{align}
Then, $V_n$ converges to the the unique fixed point $\tv$ of $\Gamma$.
\end{corollary}
\fi
Define $\tv$ denote the unique fixed point of $\Gamma$.
\todoc{Poor choice of symbols: $\tv$ is the fixed point of $\Gamma$ and $\hat{V}$ will be the fixed point of $\tilde{\Gamma}$? Hmm.. Not a high priority to change it, but still..}
\begin{lemma}\label{gfp}
The vector $\tv$ obeys $\tv\geq T\tv$.
\end{lemma}
\begin{proof}
From the definitions we have $\tv(i) = (\Gamma \tv)(i) = (\Phi r_{e_i,\tv})(i) \ge (T\tv)(i)$, $1\le i \le n$.
\if0
Consider the $i^{th}$ linear program associated with $\Gamma \tv$. We know that $\Phi r_{e_i}\geq T \tv$, for all $i=1,\ldots, n$. The result follows from noting that $\tv$ is the unique fixed point of $\Gamma $ and that $\tv(i)=(\Phi r_{e_i,V})(i)$.
\fi
\end{proof}
Recall that $\tj$ denotes the solution to the ALP in \eqref{alp}.
The fixed point of $\Gamma$ is sandwiched between $J^*$ and $\tj$:
\begin{lemma}\label{relation1}
%Assume that $\N$ is such that if $r_i$ is
%the solution to the ALP in \eqref{alp} for $c=e_i$ then $r_i\in \N$, $1\le i \le n$.
% $\tv$, the unique fixed point of the iterative scheme \eqref{pvi}, and the solution $\tj$ to the ALP in \eqref{alp}, obey 
The relation $\tj\geq\tv\geq J^*$ holds.
\end{lemma}
\begin{proof}
Since $\tv\geq T\tv$ it follows that $\tv\geq J^*$. 
Let $r_1, r_2,\ldots, r_n$ be solutions to the ALP in \eqref{alp} for $c=e_1, e_2,\ldots,e_n$ respectively
such that $r_1,\dots,r_n\in \N$. It is possible to find such solutions thanks to \cref{ass:n} \eqref{ass:n3}.
Define $V_0\in \R^n$ by $V_0(i)=\underset{j=1,\ldots, n}{\min}(\Phi r_j)(i)$, $1\le i \le n$. 
It is clear from the definition of $V_0$ that $\tj(i), \Phi r_j(i) \geq\Phi r_i(i)\geq V_0(i)$, $1\le i,j \le n$. 
Also from the monotone property of $T$, we have $T\Phi r_j\geq T V_0$,
hence we also have,
$\Phi r_j\geq T\Phi r_j \geq T V_0$, 
and so by taking component-wise minimum,
$V_0 \geq T V_0$.

Now let $V_{n+1} = \Gamma V_n$, $n=0,1,\dots$. Then, $V_1(i) = (\Gamma V_0)(i) = \min\{ V(i) \,:\, V\in \F_{V_0} \}
\le \min\{ V(i) \,:\, V\ge V_0, V = \Phi r, r\in \N \} \le \min\{ V(i)\,:\, V \ge V_0, V = \Phi r_j, 1\le j \le n \} = V_0(i)$,
where in the last step we used that $r_1,\dots,r_n \in \N$.
Hence, $V_1 \le V_0$. Using the monotone property of $\Gamma$ (i.e., \cref{gmonotone})
we then get $V_{n+1}\le V_n$ for $n\ge 0$ and thus 
$\tj\geq V_0\geq V_1\ldots\geq \tv \geq J^*$.
\end{proof}
%!TEX root =  autocontgrlp.tex
Recall that $\bj = \Gamma J^*$.
\begin{theorem}\label{fxpres}
%Let $\tv$ be the fixed point of the iterative scheme in \eqref{pvi} and let $\bj$ be the best possible LUB projection of $J^*$ as in Definition~\ref{bestproj}. Then,
We have 
\begin{align}
||J^*-\tv||_\infty\leq \frac{1}{1-\alpha}||J^*-\bj||_\infty\,.
\end{align}
\end{theorem}
\begin{proof}
Recall that $\bj = \Gamma J^*$.
Now, by the triangle inequality and since $\Gamma$ is an $\alpha$-contraction,
\begin{align*}
\inorm{J^* - \tv } \le \inorm{ J^* - \Gamma J^* } + \inorm{\Gamma J^* - \Gamma \tv }
\le  \inorm{ J^* - \bj } + \alpha \inorm{J^* - \tv}\,.
\end{align*}
Solving the inequality gives the result.
\if0
Let $\epsilon=||J^*-\bj||_\infty$, and $\{V_n\},n\geq 0$ be the iterates of the scheme in \eqref{pvi} with $V_0=\bj$, then
\begin{align}
||J^*-\tv||_\infty&\leq ||J^*-V_0+V_0-V_1+V_1\ldots-\tv||_\infty\nn\\
&\leq ||J^*-V_0||_\infty+||V_0-V_1||_\infty+\ldots\nn
\end{align}
Since $||V_1-V_0||_\infty=||\Gamma \bj-\Gamma J^*||_\infty\leq\alpha||\bj-J^*||_\infty$, from Theorem~\ref{gmaxcontra},
\begin{align}
||J^*-\tv||_\infty&\leq \epsilon+\alpha\epsilon+\alpha^2\epsilon+\ldots\nn\\
&=\frac{\epsilon}{1-\alpha}.
\end{align}
\fi
\end{proof}

