%!TEX root =  autocontgrlp.tex
\begin{comment}
\section{Approximate Linear Programming: Successes and Challenges}
When the MDP has a large number of states, it is difficult to solve for $J^*$ using either the linear program \eqref{mdplp} or other full state representation methods such as value iteration or policy iteration \cite{BertB}. A practical solution is to resort to function approximation. Linear function approximation, wherein the solution is searched in the subspace spanned by the column vectors of a given feature matrix $\Phi$.
The approximate linear program (ALP) is obtained by making use of LFA in the LP, i.e., by introducing the new variables $r\in \R^k$ and adding the extra constraint $J=\Phi r$ in \eqref{mdplp} with $\Phi \in \R^{n\times k}$ \citep{SchSei85}.
By substitution, this leads to
\begin{align}\label{alp}
\begin{split}
\min_{r\in \R^k}\, &c^\top \Phi r\,\,\,
\text{s.t.}\mb \Phi r\geq T \Phi r,
\end{split}
\end{align}
where $J\geq TJ$ is a shorthand for the $nd$ constraints in \eqref{mdplp} and $\Phi$ is a feature matrix whose first column is $\one$. Unless specified otherwise we use $\tr$ to denote an arbitrary solution to ALP, and we let $\tj=\Phi \tr$ to denote the corresponding approximate value function and $\tu$ to denote the greedy policy w.r.t. $\tj$.
The following is a preliminary error bound for ALP from \cite{ALP}:
\begin{theorem}[Error Bound for ALP]
\begin{align*}
\norm{J^*-\tj}_{1,c}\leq \frac{2}{1-\alpha}\min_{r}\norm{J^*-\Phi r}_\infty
\end{align*}
\end{theorem}
%For a more detailed treatment of ALP and sophisticated bounds, the reader is referred to \cite{ALP}.
%\subsection{Approximating the Constraints}
The ALP is a linear program in $k$ ($<<n$) variables as opposed to the LP in \eqref{mdplp} which has $n$ variables. Nevertheless, ALP has $nd$ constraints (same as the LP) which is an issue when $n$ is large and calls for constraint approximation/reduction techniques. Most works in literature make use of the underlying structure of the problem to cleverly reduce the number of constraints of ALP. A good example is \cite{gkp}, wherein the structure in factored linear functions is exploited. The use of basis function also helps constraint reduction in \cite{Mor-Kum}. In \cite{ALP-Bor}, the constraints are approximated indirectly by approximating the square of the Lagrange multipliers. In \cite{petrik} the transitional error is reduced ignoring the representational and sampling errors.\par
The most important work in the direction of constraint reduction is constraint sampling \cite{CS} wherein a reduced linear program (RLP) is solved instead of ALP. While the objective of RLP is same as that of ALP, RLP has only $m<<nd$ constraints \emph{sampled} from the original $nd$ constraints of ALP.  The following is a preliminary error bound for RLP from \cite{CS} holds for a special sampling distribution which is dependent on the optimal policy $u^*$ (see \cite{CS} for a detailed presentation):
\begin{theorem}[Error Bound for RLP]
Let $\mu_{u^*}\eqdef(1-\alpha)c^\top (I-\alpha P_{u^*})^{-1}$ be a probability distribution (of discounted number of visits) over the states $S$. Define $\psi_{u^*}$ to be the distribution amongst state-action pairs such that $\psi(s,a)\eqdef \frac{\mu_{u^*}}{d}$ and define $\theta\eqdef=\frac{1+\alpha}{2 c^\top J^*}\underset_{r\in \N}{\sup}\parallelJ^*-\Phi r \parallel_\infty$. Then for a given $\epsilon>0$ and $\delta>0$ it holds that
\begin{align*}
\norm{J^*-\Phi\tilde{r}_{RLP}}_{1,c}\leq \norm{J^*-\Phi\tilde{r}}_{1,c}+\epsilon \norm{J^*}_{1,c}
\end{align*}
for all $m\geq \frac{16d\theta}{(1-\alpha)\epsilon}\big(k\ln\frac{48d\theta}{(1-\alpha)\epsilon}+\ln \frac{2}{\delta}\big)$.
\end{theorem}
%A major gap in the theoretical analysis is that the error bounds are known for only a specific RLP formulated using idealized assumptions, i.e., under knowledge of $u^*$. However, RLP has be found to do well empirically in domains such as Tetris \cite{CST} and controlled queues \cite{CS}.
\subsection{Open Questions}
Interestingly, RLP has nevertheless been found to do well empirically in domain such as Tetris \cite{CST} and controlled queues \cite{CS} even when the constraints were sampled using distribution other than the ideal distribution. This fact indicates a gap in the theoretical analysis and points to the need for a more elaborate theory that addresses the issue of constraint approximation. In this paper, we answer the following questions related to constraint reduction in ALP that have so far remained open. \\
$\bullet$ As a natural generalization of RLP, what happens if we define a generalized reduced linear program (GRLP) whose constraints are positive linear combinations of the original constraints of ALP?\\
$\bullet$ Unlike \cite{CS} which provides error bounds for a specific RLP formulated using an idealized sampling distribution, is it possible to provide error bounds for any GRLP (and hence any RLP)?
In this paper, we address both of the questions above.
\end{comment}

%%%%%%%%%%%%%%%%%%%%%%%%%%%%%%%%%%%%%%%%%%%%%%%%
\section{Generalized Reduced Linear Programming}\label{sec:grlp}
\todoc{How about the cleaner ``Linear projected approximate linear programming''}
In this section we introduce the computational model used and the optimization problem
that is the main object of our study.
As discussed in the introduction, we are interested in methods that compute a good approximation to the optimal value function.
As noted earlier, at the expense of a modest additional cost, knowing an $O(\epsilon)$ approximation to $J^*$ at a few states suffices to compute actions of an $O(\epsilon)$-optimal policy. We will take a more general view, and we will consider calculating good approximations to $J^*$ with respect to a weighted $1$-norm, where the weights $c$ form a probability distribution over $\S$. Recall that the weighted $1$-norm $\norm{J}_{1,c}$ of a vector $J\in \R^S$ is defined as $\norm{J}_{1,c}  = \sum_s c(s) |J(s)|$. Note that here and in what follows we identify elements of $\R^\S$ (functions, mapping $\S = \{1,\dots,S\}$ to the reals) with elements of $\R^S$ in the obvious way. This allows us to write e.g. $c^\top J$, which denotes $\sum_s c(s) J(s)$.

To introduce the optimization problem we study first recall that 
the optimal value function $J^*$ is the solution of the fixed point equation $TJ^* = J^*$. 
It follows from the definition of $T$ that $J^* = \max_u T_u J^* \ge T_u J^*$ for any $u$, 
where $\ge$ is the componentwise partial ordering of vectors. 
With some abuse of notation, we also introduce $T_a$ to denote $T_u$ where $u(s) = a$ for any $s\in \S$. 
It follows that $J^* \ge T_a J^*$ for any $a\in \A$ and also that $T = \max_a T_a$, where again the maximization is componentwise.
We call a vector $J$ that satisfies $J \ge T_a J$ for any $a\in A$ \emph{superharmonic}. Note that this is a set of linear inequalities. 
By our note on $T$ and $(T_a)_a$, these inequalities can also be written compactly as $J \ge T J$.
\if0
Since $T$ is monotone (i.e., for any $J_1\le J_2$ it holds that $TJ_1 \le T J_2$) and can be seen to be an $\alpha$-contraction with respect to the maximum norm, if $J$ is superharmonic then $J \ge T J \ge T^2 J \ge \dots \ge J^*$, where the last inequality follows since $T$ is an $\alpha$-contraction and hence $T^n J \to J^*$ in the maximum norm, and hence also componentwise.
Thus, $J^*$ is the ``smallest'' superharmonic function. It follows than that 
for any $c\in \R_+^S \doteq [0,\infty)^S$ and any superharmonic $J$, $c^\top J^*\le c^\top J$ and since $J^*$ is also superharmonic, $\min \{ c^\top J\,:\, J \ge T J \} = c^\top J^*$ and if $c$ is positive-valued then the minimizer of $c^\top J$ amongst all superharminoc functions is $J^*$.
\fi
It is not hard to show then that $J^*$ is the smallest superharminoc function (i.e., for any $J$ superharmonic, $J\ge J^*$). It also follows that for any  $c\in \R_{++}^S \doteq (0,\infty)^S$, the unique solution to the linear program $\min\{ c^\top J \,:\, J \ge T J \} =\min \{ c^\top J\,:\, J \ge T_a J, a\in \A \} $ is $J^*$. 

Now, let $\phi_1,\ldots,\phi_k : \S \to \R$ be $k$ basis functions. 
\todoc{I think we should use $d$ to denote the number of basis functions. Or at least $n$. Low priority.} 
The \emph{Approximate Linear Program} (ALP) of \citet{SchSei85} 
is obtained by adding the linear constraints $J = \sum_{i=1}^k r_i \phi_i$ to the above linear program. Eliminating $J$ then gives $\min\{ \sum_i r_i c^\top \phi_i \,:\, \sum_i r_i \phi_i \ge g_a + \alpha \sum_i r_i P_a \phi_i, a\in \A, r = (r_i)\in \R^k \}$.
As noted by \citet{SchSei85}, the linear program is feasible as long as $\one$, defined as the vector with all components being identically equal to one, is in the span of $\{\phi_1,\dots,\phi_k\}$. 
\emph{For the purpose of computations, it is assumed that the values $c^\top \phi_i$, $i=1,\dots, k$ and the values $(P_a \phi_i)(s)$ and $g_a(s)$ can be accessed in constant time.} 
This assumption can be relaxed to assuming that one can access $g_a(s)$ and $\phi_i(s)$ for any $(s,a)$ in constant time, as well as to that one can efficiently sample from $c$, from $P_a(s,\cdot)$ for any $(s,a)$ pair, 
but the details of this are the beyond the scope of the present work. As shown by \citet{ALP}, if $\ralp$ denotes the solution to the above ALP then for $\Jalp \doteq \sum_i \ralp(i) \phi_i \doteq \Phi r$ it holds that $\norm{\Jalp - J^*}_{1,c} \le \frac{2 \epsilon}{1-\alpha}$ where $\epsilon = \inf_r \norm{ J^* - \Phi r }_\infty$ is the error of approximating the optimal value with the span of the basis functions $\phi_1,\dots,\phi_k$ and $\norm{J}_\infty = \max_s |J(s)|$ is the maximum norm and $\Phi \in \R^{S\times k}$ is the matrix formed by $(\phi_1,\dots,\phi_k)$.

While evaluating the objective and checking a single constraint for constraint violations can be done in constant time, 
since the number of constraints in the ALP is equal to the number of state-action pairs, the ALP is still intractable in general.
To reduce the number of constraints, we propose to relax the constraints by considering a linear projection of them with 
positive linear coefficients. Since the resulting program may become unbounded, it is necessary to introduce an additional constraint $r\in \N$, where $\N$ will be chosen to be large enough so as not to exclude ``good solutions'', while constraining enough so as to prevent unbounded solutions. \todoc{I still don't get whether we need or not $\N$. Under the conic constraints, do we need it or not? I mean when $r$ hits the boundary of $\N$ I have the feeling that we get a very bad solution and it would not then matter much whether $\N$ is even there or not.}
This leads to what we call the Generalized Reduced Linear Program: \todoc{Again, Linearly Projected ALP, or ALP with Linear Constraint Projection, or with Linear Constrain Relaxation?} \todoc{I stopped here, need to sleep. The idea in this section is to just explain how the computation is done. In the next section I would just give the main result. In the next section I would give the proof. Rather than the present lemma, lemma, lemma.. organization it is probably better to organize the proof in a top-down. Starting to do the proof, required results can be stated as claims/lemmas. But maybe some of the lemmas can be saved (I have a specific plan, just saying, if you want to do something with this).}
\begin{align}\label{grlp}
\begin{split}
&\underset{r\in \N\subset R^k}{\min}\, \, c^\top \Phi r\,\,\,\,\\
&\text{s.t.}\mb  \,W^\top E\Phi r\geq W^\top H \Phi r,
\end{split}
\end{align}
where we assume that $W \in \R^{nd\times m}_+$ is a matrix with all positive entries which specifies the linear combination of constraints.  We denote an arbitrary solution to GRLP by $\hr$, and the approximate value function by $\hj=\Phi \hr$ and use $\hu$ to denote the greedy policy w.r.t. $\hj$.\\
\textbf{RLP as a special case:} Recovering RLP \eqref{rlp} as a special case of GRLP \eqref{grlp} can be accomplished by setting $W$ to the matrix with the $i^{th}$ column having $1$ corresponding to the constraint that was sampled (implicitly assuming that there is a natural ordering of the $nd$ constraints) and all the other entries as $0$. The $\N$ in RLP \eqref{rlp} is retained in GRLP \eqref{grlp} as well. For $m=nd$ and $W=I_{nd\times nd}$, GRLP is same as ALP.\\
\textbf{Linear function approximation in Primal and Dual Variables:} Let us look at the Lagrangian of ALP and GRLP in
\eqref{lag} and \eqref{lag2} respectively, i.e.,
\begin{align}\label{lag}
\tilde{L}(r,\lambda)=c^\top \Phi r+\lambda^\top (T\Phi r-\Phi r), \\ \label{lag2}\hat{L}(r,q)=c^\top \Phi r+q^\top W^\top (T\Phi r-\Phi r).
\end{align}
Thus, when $Wq = \lambda$, i.e., when $W$ is a set of basis functions that allow
a low dimensional linear representation of the dual variables $\lambda$,
the two problems are the same.
%Note that $ Wq\approx \lambda$ in \eqref{lag2} is linear function approximation of the Lagrange multipliers.
Hence, while ALP employs LFA in its objective function (i.e., use of $\Phi r$), GRLP employs linear approximation both in the objective function ($\Phi r$) as well as the constraints (use of $W$).
%Further, $W$ can be interpreted as the feature matrix that approximates the Lagrange multipliers as $\lambda\approx Wq$, where $\lambda \in \R^{nd}, r\in \R^m$.
To get a sense of how $W$ should be chosen, recall that
the optimal Lagrange multipliers are the discounted number of visits to the ``state-action pairs'' under an optimal policy $u^*$, i.e.,
\begin{align}
\lambda^*(s,u^*(s))&=\big(c^\top(I-\alpha P_{u^*})^{-1}\big)(s)\nn\\
&= \big(c^\top(I+\alpha P_{u^*}+\alpha^2 P_{u^*}^2+\ldots)\big)(s),\nn\\
\lambda^*(s,a)&=0, \qquad \text{for all } a \neq u^*(s),\nn
\end{align}
where $P_{u^*}$ is the probability transition matrix under $u^*$ ($P_{u^*}(s,s') = P_{u^*(s)}(s,s')$, $s,s'\in S$) \cite{dolgov}. Even though we might not have the optimal policy $u^*$ in practice, the fact that $\lambda^*$ is a probability distribution and that it is a linear combination of $\{P_{u^*},P^2_{u^*},\ldots\}$ hints at the kind of features that might be useful for constructing the $W$ matrix.
\FloatBarrier
%%!TEX root =  autocontgrlp.tex
\begin{figure}[h!]
\centering
%\resizebox{columnwidth}{2cm}{
%\begin{tikzpicture}[domain=-10:7.7,scale=0.6,font=\small,axis/.style={very thick, ->, >=stealth'}]
\begin{tikzpicture}[domain=-10:7.7,scale=0.5,font=\normalsize,axis/.style={very thick, ->, >=stealth'}]
%\draw[line,thick,->] (-1,-0.625)--(4,-0.625);
%\draw[line,thick,->] (0,-1)--(0,4);
%\draw[line,thick,-](-0.2,-0.6)--(1,3);
\draw[line,thick,-](0.5,3.5)--(1,1);
\draw[line,thick,-](1,1)--(3,1.5);
\draw[line,thick,-](3,1.5)--(4,2.1);
\node[](one) at (2,2.3){\text{$J\geq TJ$}};
\node[rotate=-45](seven) at (-0.5,2.3){\text{\tiny $W^\top J\geq W^\top TJ$}};
\node[](two) at (-0.3,-0.5){\text{$\Phi r$}};
\node[](three) at (0.7,0.7){\text{$J^*$}};
%\draw[line,thick,-](0,0)--(4,2.5);
 \draw [ultra thick, draw=white, fill=gray, opacity=0.5]
       (0.5,3.5)--(1,1)--(3,1.5)--(4,2.1) -- cycle;
\draw[line,thick,-](-1,-0.625)--(3,1.8750);
 \fill (1,1)  circle[radius=2pt];
 \fill (2,1.25)  circle[radius=2pt];
 \fill (0.5,0.3125)  circle[radius=2pt];
\draw[line,dashed,-](-1,0.1)--(6,1);
\draw[line,dashed,-](-1,0.1)--(-2,5);
 \draw [ultra thick, draw=white, fill=gray, opacity=0.2]
       (-1,0.1)--(6,1)--(-2,5) -- cycle;
\node[] (four) at (2,0.8){\text{$\tilde{J}_c$}};
\node[] (six) at (0.5,-0.1){\text{$\hat{J}_c$}};
\end{tikzpicture}
%}
\caption{
%\normalsize 
The outer lightly shaded region corresponds to GRLP constraints and the inner dark shaded region corresponds to the original constraints. The main contribution of the paper is to bound $||J^*-\hat{J}_c||$.}
\label{cartoon}
\end{figure}


\begin{figure}
\includegraphics[scale=0.7]{cartoon_grlp.pdf}
\caption{
%\normalsize
The outer lightly shaded region corresponds to GRLP constraints and the inner dark shaded region corresponds to the original constraints. The main contribution of the paper is to bound $\norm{J^*-\hat{J}_c}$.}
\label{cartoon}
\end{figure}
\Cref{cartoon} shows the solutions to the LP, ALP and GRLP respectively. The error in ALP solution has already been studied in \cite{ALP}. Our objective is to study the extra source of error due to constraint approximation.
\section{Main Results}
In this section we present the main results of this paper in \Cref{cmt2} (we state improved bounds in \Cref{sec:improv}). Our bounds are expressed in terms of two novel contractions operators which we define in \Cref{lubpop,alubpop}. We now define two projection operators that are central to our error analysis and in them we assume that the set $\N'\subset \R^k$ is such that $\N' = \N + t \one$ for any $t\in \R$.
%The least upper bound (LUB) projection operator $\Gamma \colon \R^n \ra\R^n$ is defined below, see \eqref{gamdef}.
\begin{definition}\label{lubpop}
Given $J\in \R^n$ and the nonnegative valued vector $c\in \R^n_+$, define $r_{c,J}$ to be the solution to
\begin{align}
\label{lubplp}
\begin{split}
\underset{r\in \N'}{\min} &\,\, c^\top \Phi r\,\mb
\text{s.t.} \mb \Phi r\geq  TJ.
\end{split}
\end{align}
For $J\in \R^n$, $\Gamma J$, the \emph{least upper} (LU) projection of $J$ is defined as
\begin{align}\label{gamdef}
(\Gamma J)(i)\eqdef(\Phi r_{e_i,J})(i),\quad i=1,\ldots,n\,.
\end{align}
\end{definition}
The definition of the second operator is as follows:
\begin{definition}\label{alubpop}
Given $J\in \R^n$ and the nonnegative valued vector $c\in \R^n_+$, define $r'_{c,J}$ to be the solution to
\begin{align}\label{alubplp}
\underset{r\in \N'}{\min}& \,\mb c^\top \Phi r\,\mb
\text{s.t.} \,\,\, W^\top E \Phi r\geq W^\top HJ.
\end{align}
The \emph{approximate least upper} (ALU) projection operator
$\hg \colon \R^n \ra \R^n$ is defined as
\begin{align}\label{tgamdef}
(\hg J)(i)\eqdef(\Phi r'_{e_i,J})(i), \mb i=1,\ldots,n\,, J\in \R^n\,.
\end{align}
\end{definition}
\begin{remark}\label{ubrem}
To understand the meaning of $\Gamma$ (and $\hg$) define
\begin{align}\label{ubclass}
\F_J\eqdef\{\,\Phi r\,:\,\Phi r\geq TJ, r\in \N'\,\},
\end{align}
where $J\in \R^n$.
Disregarding the constraint $r\in \N'$,
$\F_J$ contains all vectors in the span of $\Phi$ that upper bound $TJ$. Further, since $(\Gamma J)(i) = \min\{ V(i) \,:\, V\in \F_J \}$, it also follows that $ V\ge \Gamma J $ holds for any $V\in \F_J$. Also $\Gamma J\geq T J$.\par
The operators $\Gamma$ and $\hg$ are closely related to ALP \eqref{alp} and GRLP \eqref{grlp} respectively and are only analytical tools that we will need to express our bounds and need not be computed in practice. It turns out that the novel operators ($\Gamma$ and $\hg$) are contraction maps (see \Cref{hgmaxcontramn}), a fact that is a key to our results (\Cref{cmt2mn,polthe}). At the outset we are interested in the candidate value functions in the constraint set $\N$ and want to study them using $\Gamma$ and $\hg$. However since the basis $\Phi$ contains $\one$ we make it helps our analysis to define the set $\N'$ in \Cref{lubplp,alubplp} to be `$\N$ plus its translations by $\one$'.
\end{remark}
%\subsection{Error Bounds}
\begin{theorem}[Error Bound for GRLP]
\label{cmt2}
Let $\one\in\R^k$ be in the column span of $\Phi$, then it holds that
\begin{enumerate}
\item The prediction error is bound as\\
$\norm{J^*-\hj}_{1,c}
\leq\frac{1}{1-\alpha}(6~\us{\min}{r\in \R^k}\norm{J^*-\Phi r}_{\infty}
+2\norm{\Gamma J^*-\hg J^*}_{\infty})$.
\item The control error is bound as\\
$\norm{J^* - J_{\hu}}_{1,c}
\leq 2\left(\frac{1}{(1-\alpha)^2}\right)\, \big( 2~\us{\min}{r\in \R^k} \norm{J^*-\Phi r}_{\infty}
+\norm{\Gamma J^*-\hg J^*}_{\infty}+\norm{\hj-\hg\hj}_{\infty}\big)$.
\end{enumerate}
\end{theorem}
\textbf{On Prediction Error:} The first factor in the right hand side of the prediction error in \Cref{cmt2} is related to the best possible approximation that can be achieved with the chosen feature matrix $\Phi$. This term is an carry over of the upper bound in ALP formulation as shown in \Cref{alpvanilla}. The second factor in the right hand side of the prediction error is related to constraint approximation and is completely defined in terms of $\Phi$, $W$ and $T$, and does not require knowledge of stationary distribution of the optimal policy.\par
\textbf{On Control Error:} The first two terms are quite similar to those in the bound for prediction error. The third term occurs due to the fact that $\Phi \geq T\Phi r$ that holds in the case of ALP does not hold in the case of GRLP.\par
\begin{comment}
\begin{theorem}[Control Error Bound in $\norm{\cdot}_{\infty}$]
\label{polthe}
Let $\hu$ be the greedy policy with respect to the solution $\hj$ of GRLP and $J_{\hu}$ be its value function.
% Let $r^*$ be as in Theorem~\ref{mt2mn}, then
Then,
\begin{align}\label{polthebnd}
\norm{J^* - J_{\hu}}_{1,c}
&\leq 2\left(\frac{c^\top \psi}{(1-\beta_{\psi})^2}\right)\, \big( 2\norm{J^*-\Phi r^*}_{\infty}
\nn\\&
+norm{\Gamma J^*-\hg J^*}_{\infty}+\norm{\hj-\hg\hj}_{\infty}\big).
\end{align}
\end{theorem}
\end{comment}
\begin{corollary}[Constraint Sampling]\label{st}
$W\in \{0,1\}^{nd\times m}$Let $s\in S$ be a state whose constraint is selected by $W$ (i.e., for some $i$ and all $(s',a)\in S\times A$,
$W_{s'a,i}=\delta_{s=s'}$.
Then
\begin{align*}%\label{sampexp}
|\Gamma J^*(s)-\hg J^*(s)|<|\Gamma J^*(s)-J^*(s)|.
\end{align*}
\end{corollary}
It is important to note that \Cref{rlpt} holds only in high probability and is valid only under idealized assumption of knowing the optimal policy $u^*$, while \Cref{cmt2} does not have these limitations.
In addition, the error $|\Gamma J^*(s) -\hg J^*(s)|$ (in \Cref{st} ) due to constraint approximation is less than the original projection error $|\Gamma J^*(s)-J^*(s)|$ due to function approximation. This means that for RLP to perform well it is important to retain the constraints corresponding to those states for which the linear function approximation via $\Phi$ is known to perform well.
