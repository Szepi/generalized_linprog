%!TEX root =  autocontgrlp.tex
\begin{comment}
\section{Approximate Linear Programming: Successes and Challanges}
When the MDP has a large number of states, it is difficult to solve for $J^*$ using either the linear program \eqref{mdplp} or other full state representation methods such as value iteration or policy iteration \cite{BertB}. A practical solution is to resort to function approximation. Linear function approximation, wherein the solution is searched in the subspace spanned by the column vectors of a given feature matrix $\Phi$.
The approximate linear program (ALP) is obtained by making use of LFA in the LP, i.e., by introducing the new variables $r\in \R^k$ and adding the extra constraint $J=\Phi r$ in \eqref{mdplp} with $\Phi \in \R^{n\times k}$ \citep{SchSei85}.
By substitution, this leads to
\begin{align}\label{alp}
\begin{split}
\min_{r\in \R^k}\, &c^\top \Phi r\,\,\,
\text{s.t.}\mb \Phi r\geq T \Phi r,
\end{split}
\end{align}
where $J\geq TJ$ is a shorthand for the $nd$ constraints in \eqref{mdplp} and $\Phi$ is a feature matrix whose first column is $\one$. Unless specified otherwise we use $\tr$ to denote an arbitrary solution to the ALP, and we let $\tj=\Phi \tr$ to denote the corresponding approximate value function and $\tu$ to denote the greedy policy w.r.t. $\tj$.
The following is a preliminary error bound for the ALP from \cite{ALP}:
\begin{theorem}[Error Bound for ALP]
\begin{align*}
||J^*-\tj||_{1,c}\leq \frac{2}{1-\alpha}\min_{r}||J^*-\Phi r||_\infty
\end{align*}
\end{theorem}
%For a more detailed treatment of the ALP and sophisticated bounds, the reader is referred to \cite{ALP}.
%\subsection{Approximating the Constraints}
The ALP is a linear program in $k$ ($<<n$) variables as opposed to the LP in \eqref{mdplp} which has $n$ variables. Nevertheless, the ALP has $nd$ constraints (same as the LP) which is an issue when $n$ is large and calls for constraint approximation/reduction techniques. Most works in literature make use of the underlying structure of the problem to cleverly reduce the number of constraints of the ALP. A good example is \cite{gkp}, wherein the structure in factored linear functions is exploited. The use of basis function also helps constraint reduction in \cite{Mor-Kum}. In \cite{ALP-Bor}, the constraints are approximated indirectly by approximating the square of the Lagrange multipliers. In \cite{petrik} the transitional error is reduced ignoring the representational and sampling errors.\par
The most important work in the direction of constraint reduction is constraint sampling \cite{CS} wherein a reduced linear program (RLP) is solved instead of the ALP. While the objective of the RLP is same as that of the ALP, the RLP has only $m<<nd$ constraints \emph{sampled} from the original $nd$ constraints of the ALP.  The following is a preliminary error bound for the RLP from \cite{CS} holds for a special sampling distribution which is dependent on the optimal policy $u^*$ (see \cite{CS} for a detailed presentation):
\begin{theorem}[Error Bound for RLP]
If $\tilde{r}_{RLP}$ is a solution to the RLP, then
\begin{align*}
||J^*-\Phi\tilde{r}_{RLP}||_{1,c}\leq ||J^*-\Phi\tilde{r}||_{1,c}+\epsilon ||J^*||_{1,c}
\end{align*}
\end{theorem}
%A major gap in the theoretical analysis is that the error bounds are known for only a specific RLP formulated using idealized assumptions, i.e., under knowledge of $u^*$. However, the RLP has be found to do well empirically in domains such as Tetris \cite{CST} and controlled queues \cite{CS}.
\subsection{Open Questions}
Interestingly, the RLP has nevertheless been found to do well empirically in domain such as Tetris \cite{CST} and controlled queues \cite{CS} even when the constraints were sampled using distribution other than the ideal distribution. This fact indicates a gap in the theoretical analysis and points to the need for a more elaborate theory that addresses the issue of constraint approximation. In this paper, we answer the following questions related to constraint reduction in ALP that have so far remained open. \\
$\bullet$ As a natural generalization of the RLP, what happens if we define a generalized reduced linear program (GRLP) whose constraints are positive linear combinations of the original constraints of the ALP?\\
$\bullet$ Unlike \cite{CS} which provides error bounds for a specific RLP formulated using an idealized sampling distribution, is it possible to provide error bounds for any GRLP (and hence any RLP)?
In this paper, we address both of the questions above.
\end{comment}
\section{Main Results}
\begin{comment}
The Generalized Reduced Linear Program is given as:
\begin{align}\label{grlp}
\begin{split}
&\underset{r\in \N\subset R^k}{\min}\, \, c^\top \Phi r\,\,\,\,\\
&\text{s.t.}\mb  \,W^\top E\Phi r\geq W^\top H \Phi r,
\end{split}
\end{align}
where we assume that $W \in \R^{nd\times m}_+$ is a matrix with all positive entries, $\N$ is an additional (compact) constraint set to ensure the boundedness of the solution and $\tr \in \N$.
\footnote{The appendix explains how $\N$ can be chosen.}
In what follows, we denote the solution to the GRLP by $\hr$, the approximate value function by $\hj=\Phi \hr$ and use $\hu$ to denote the greedy policy w.r.t. $\hj$.\par
\end{comment}
%\subsection{Novel Projection Operators}
We now define two projection operators which are central to our error analysis and in them we assume that the set $\N'\subset \R^k$ is such that $\N' = \N + t \one$ for any $t\in \R$.
%The least upper bound (LUB) projection operator $\Gamma \colon \R^n \ra\R^n$ is defined below, see \eqref{gamdef}.
\begin{definition}\label{lubpop}
Given $J\in \R^n$ and the nonnegative valued vector $c\in \R^n_+$, define $r_{c,J}$ to be the solution to
\begin{align}
\label{lubplp}
\begin{split}
\underset{r\in \N'}{\min} &\,\, c^\top \Phi r\,\mb
\text{s.t.} \mb \Phi r\geq  TJ.
\end{split}
\end{align}
Then, for $J\in \R^n$, $\Gamma J$,
the \emph{ least upper bound projection} of $J$ is defined as
\begin{align}\label{gamdef}
(\Gamma J)(i)\eqdef(\Phi r_{e_i,J})(i),\quad i=1,\ldots,n\,.
\end{align}
\end{definition}

%The approximate least upper bound (ALUB) projection operator $\hg \colon \R^n \ra \R^n$ is defined as below:
The definition of the second operator is as follows:
\begin{definition}\label{alubpop}
Given $J\in \R^n$ and the nonnegative valued vector $c\in \R^n_+$, define $r'_{c,J}$ to be the solution to
\begin{align}\label{alubplp}
\underset{r\in \N'}{\min}& \,\mb c^\top \Phi r\,\mb
\text{s.t.} \,\,\, W^\top E \Phi r\geq W^\top HJ.
\end{align}
Then, the \emph{approximate least upper bound (ALUB) projection operator}
$\hg \colon \R^n \ra \R^n$ is defined as
\begin{align}\label{tgamdef}
(\hg J)(i)\eqdef(\Phi r'_{e_i,J})(i), \mb i=1,\ldots,n\,, J\in \R^n\,.
\end{align}
\end{definition}
\begin{remark}\label{ubrem}
To understand the meaning of $\Gamma$ (and $\hg$) define
\begin{align}\label{ubclass}
\F_J\eqdef\{\,\Phi r\,:\,\Phi r\geq TJ, r\in \N\,\},
\end{align}
where $J\in \R^n$.
Disregarding the constraint $r\in \N$,
$\F_J$ contains vectors in the span of $\Phi$ that upper bound $TJ$. Further, since $(\Gamma J)(i) = \min\{ V(i) \,:\, V\in \F_J \}$, it also follows that $ V\ge \Gamma J $ holds for any $V\in \F_J$.
\end{remark}

%%!TEX root =  autocontgrlp.tex
\begin{figure}[h!]
\centering
%\resizebox{columnwidth}{2cm}{
%\begin{tikzpicture}[domain=-10:7.7,scale=0.6,font=\small,axis/.style={very thick, ->, >=stealth'}]
\begin{tikzpicture}[domain=-10:7.7,scale=0.5,font=\normalsize,axis/.style={very thick, ->, >=stealth'}]
%\draw[line,thick,->] (-1,-0.625)--(4,-0.625);
%\draw[line,thick,->] (0,-1)--(0,4);
%\draw[line,thick,-](-0.2,-0.6)--(1,3);
\draw[line,thick,-](0.5,3.5)--(1,1);
\draw[line,thick,-](1,1)--(3,1.5);
\draw[line,thick,-](3,1.5)--(4,2.1);
\node[](one) at (2,2.3){\text{$J\geq TJ$}};
\node[rotate=-45](seven) at (-0.5,2.3){\text{\tiny $W^\top J\geq W^\top TJ$}};
\node[](two) at (-0.3,-0.5){\text{$\Phi r$}};
\node[](three) at (0.7,0.7){\text{$J^*$}};
%\draw[line,thick,-](0,0)--(4,2.5);
 \draw [ultra thick, draw=white, fill=gray, opacity=0.5]
       (0.5,3.5)--(1,1)--(3,1.5)--(4,2.1) -- cycle;
\draw[line,thick,-](-1,-0.625)--(3,1.8750);
 \fill (1,1)  circle[radius=2pt];
 \fill (2,1.25)  circle[radius=2pt];
 \fill (0.5,0.3125)  circle[radius=2pt];
\draw[line,dashed,-](-1,0.1)--(6,1);
\draw[line,dashed,-](-1,0.1)--(-2,5);
 \draw [ultra thick, draw=white, fill=gray, opacity=0.2]
       (-1,0.1)--(6,1)--(-2,5) -- cycle;
\node[] (four) at (2,0.8){\text{$\tilde{J}_c$}};
\node[] (six) at (0.5,-0.1){\text{$\hat{J}_c$}};
\end{tikzpicture}
%}
\caption{
%\normalsize 
The outer lightly shaded region corresponds to GRLP constraints and the inner dark shaded region corresponds to the original constraints. The main contribution of the paper is to bound $||J^*-\hat{J}_c||$.}
\label{cartoon}
\end{figure}


\subsection{Error Bounds}
\begin{theorem}[Error Bound for GRLP]
\label{cmt2}
%Let $\hj$, $\hv$, $r^*$ and $J^*$ be as in Theorem~\ref{mt2mn}, then
It holds that
\begin{align}\label{finalbnd}
\begin{split}
||J^*-\hj||_{1,c}
\leq\frac{1}{1-\alpha}(6 ||J^*-\Phi r^*||_{\infty}
+2||\Gamma J^*-\hg J^*||_{\infty}).
\end{split}
\end{align}
\end{theorem}
\begin{comment}
\begin{theorem}[Control Error Bound in $\norm{\cdot}_{\infty}$]
\label{polthe}
Let $\hu$ be the greedy policy with respect to the solution $\hj$ of the GRLP and $J_{\hu}$ be its value function.
% Let $r^*$ be as in Theorem~\ref{mt2mn}, then
Then,
\begin{align}\label{polthebnd}
%||J_{\hu}-\hj||_{1,c}
\norm{J^* - J_{\hu}}_{1,c}
&\leq 2\left(\frac{c^\top \psi}{(1-\beta_{\psi})^2}\right)\, \big( 2||J^*-\Phi r^*||_{\infty}
\nn\\&
+||\Gamma J^*-\hg J^*||_{\infty}+||\hj-\hg\hj||_{\infty}\big).
\end{align}
\end{theorem}
\end{comment}
\begin{theorem}[Constraint Sampling]\label{st}
Let $s\in S$ be a state whose constraint is selected by $W$ (i.e., for some $i$ and all $(s',a)\in S\times A$,
$W_{s'a,i}=\delta_{s=s'}$.
Then
\begin{align*}%\label{sampexp}
|\Gamma J^*(s)-\hg J^*(s)|<|\Gamma J^*(s)-J^*(s)|.
\end{align*}
\end{theorem}
\begin{comment}
An interesting feature is that unlike prior work on constraint sampling based on concentration inequalities (e.g.,  \cite{CS}), our analysis is based on contraction operators and is completely deterministic.
In particular, the error term $\etmn$ gives new insights into constraint selection:
The result of Theorem~\ref{st} is that
\end{comment}
%The modified $L_\infty$ norm in \eqref{finalbndmn} comes to our rescue to control the error due to those states that are not chosen.
