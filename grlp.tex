%!TEX root =  autocontgrlp.tex
\section{Generalized Reduced Linear Program}
The approximate linear program (ALP) is obtained by making use of LFA in the LP, i.e., by introducing the new variables $r\in \R^k$ and adding the extra constraint $J=\Phi r$ in \eqref{mdplp} with $\Phi \in \R^{n\times k}$ \citep{SchSei85}. 
By substitution, this leads to
\begin{align}\label{alp}
\begin{split}
\min_{r\in \R^k}\, &c^\top \Phi r\,\,\,
\text{s.t.}\mb \Phi r\geq T \Phi r,
\end{split}
\end{align}
where $J\geq TJ$ is a shorthand for the $nd$ constraints in \eqref{mdplp}. Unless specified otherwise we use $\tr$ to denote an arbitrary solution to the ALP, and we let $\tj=\Phi \tr$ to denote the corresponding approximate value function and $\tu$ to denote the greedy policy w.r.t. $\tj$.
The Generalized Reduced Linear Program is given as:
\begin{align}\label{grlp}
\begin{split}
\underset{r\in \N}{\min}\, &\, c^\top \Phi r\,\,\,\,
\text{s.t.}\mb  \,W^\top E\Phi r\geq W^\top H \Phi r,
\end{split}
\end{align}
where $W \in \R^{nd\times m}_+$ is a matrix with all positive entries and $\N$ is an additional (compact) constraint set to ensure the boundedness of the solution.%
\footnote{The appendix explains how $\N$ can be chosen.}
In what follows, we denote the solution to the GRLP by $\hr$, the approximate value function by $\hj=\Phi \hr$ and use $\hu$ to denote the greedy policy w.r.t. $\hj$.

We assume that the following hold in the rest of the paper:
\begin{assumption}\label{grlpassmp}
\begin{enumerate}[(i)]
\item \label{probdist} $c=(c(i),i=1,\ldots,n)\in \R^n$ is a positive probability distribution, i.e., $c(i)>0 \mb \forall i$ and $\sum_{i=1}^n c(i)=1$.
\item  \label{one} The first column of the feature matrix $\Phi$ (i.e., $\phi_1$) is $\one \in \R^n$. 
\item \label{lyap} $\psi\colon S \ra \R_+$ is a Lyapunov function for $P$
and is present in the column span of the feature matrix $\Phi$: For some $r_0\in \R^k$, $\Phi r_0 = \psi$. 
\item \label{nassmp} $\N\subset \R^k$ is compact and $\tr \in \N\subset\R^k$.
\item \label{wassump} $W \in \R^{nd\times m}_+$ is a full rank $nd\times m$ matrix (where $m\ll nd$), with all non-negative entries such that each of its column-sums equals one.
\item \label{ass:n4} The set $\N'$ is such that $\N' = \N + t r_0$ for any $t\in \R$, where $r_0\in \R^k$ such that $\Phi r_0 = \psi$.
\end{enumerate}
\end{assumption}
We note in passing that if \cref{grlpassmp}-\eqref{lyap} holds, it follows that for any $J\in \R^n$ and $t>0$,
\begin{align}
\label{eq:psilin}
T(J+ t \psi ) \le TJ + \beta_{\psi}\,t\,  \psi\,. %\,\, \quad (J\in \R^n,\, t>0)\,.
\end{align}

The GRLP introduces linear function approximation in both the primal and dual variables of the LP formulation. 
To understand this, we need to look at the Lagrangian of the ALP and GRLP in 
\eqref{lag} and \eqref{lag2} respectively, i.e., 
\begin{align}\label{lag}
\tilde{L}(r,\lambda)=c^\top \Phi r+\lambda^\top (T\Phi r-\Phi r), \\ \label{lag2}\hat{L}(r,q)=c^\top \Phi r+q^\top W^\top (T\Phi r-\Phi r).
\end{align}
Thus, when $Wq = \lambda$, i.e., when $W$ is a set of basis functions that allow
a low dimensional linear representation of the dual variables $\lambda$,
the two problems are the same.
%Note that $ Wq\approx \lambda$ in \eqref{lag2} is linear function approximation of the Lagrange multipliers. 
Hence, while the ALP employs LFA in its objective function (i.e., use of $\Phi r$), the GRLP employs linear approximation both in the objective function ($\Phi r$) as well as the constraints (use of $W$). 
%Further, $W$ can be interpreted as the feature matrix that approximates the Lagrange multipliers as $\lambda\approx Wq$, where $\lambda \in \R^{nd}, r\in \R^m$. 
To get a sense of how $W$ should be chosen, recall that
the optimal Lagrange multipliers are the discounted number of visits to the ``state-action pairs'' under an optimal policy $u^*$, i.e., 
\begin{align}
\lambda^*(s,u^*(s))&=\big(c^\top(I-\alpha P_{u^*})^{-1}\big)(s)\nn\\
				&= \big(c^\top(I+\alpha P_{u^*}+\alpha^2 P_{u^*}^2+\ldots)\big)(s),\nn\\
			\lambda^*(s,a)&=0, \qquad \text{for all } a \neq u^*(s),\nn
\end{align}
where $P_{u^*}$ is the probability transition matrix under $u^*$ ($P_{u^*}(s,s') = P_{u^*(s)}(s,s')$, $s,s'\in S$) \cite{dolgov}. Even though we might not have the optimal policy $u^*$ in practice, the fact that $\lambda^*$ is a probability distribution and that it is a linear combination of $\{P_{u^*},P^2_{u^*},\ldots\}$ hints at the kind of features that might be useful for the $W$ matrix.
